%%%%%%%%%%%%%%%%%%%%% chapter.tex %%%%%%%%%%%%%%%%%%%%%%%%%%%%%%%%%
%
% sample chapter
%
% Use this file as a template for your own input.
%
%%%%%%%%%%%%%%%%%%%%%%%% Springer-Verlag %%%%%%%%%%%%%%%%%%%%%%%%%%

\chapstarthook{The content of this appendix has been submitted to
\emph{Data \& Knowledge Engineering} [IF2007: 1.144].}

\chapter{A Survey on Summarizability Issues in Multidimensional Modeling}
\label{a1} % Always give a unique label
% use \chaptermark{}
% to alter or adjust the chapter heading in the running head

\chaptermark{A survey on summarizability issues}


The development of a data warehouse (DW) system is based on a
conceptual multidimensional (MD) model, which provides a high level
of abstraction in accurately and expressively describing real-world
situations. Once this model is designed, the corresponding logical
representation must be obtained as the basis of the implementation
of the DW according to one specific technology. However, even though
a good conceptual MD model is designed underneath a DW, there is a
semantic gap between this model and its logical representation. In
particular, this gap complicates an adequate treatment of
summarizability issues, which in turn may lead to erroneous results
of data analysis tools. Research addressing this topic has produced
only partial solutions, and individual terminology used by different
parties hinders further progress. Consequently, based on a unifying
vocabulary, this survey sheds light on (i) the weak and strong
points of current approaches for modeling complex MD structures that
reflect real-world situations in a conceptual MD model and (ii)
existing mechanisms to avoid summarizability problems when
conceptual MD models are being implemented.



\section{Introduction}
\label{a1:sec:intro} Nowadays, data warehouse (DW) systems play a
decisive role in providing companies with many years of historical
information in an accurate way for the decision making process. The
development of these systems is based on multidimensional (MD)
modeling~\cite{book/Jarke/DW}, since (i) it is close to the way of
thinking of human analysts and, therefore, it helps users to
understand data; and (ii) it supports performance improvements as
its simple structure allows designers to predict decision makers'
intentions.

MD models structure data according to a multidimensional space,
where dimensions specify different ways the data can be viewed,
aggregated, and sorted (e.g., according to time, store, customer,
product, etc.).  Events of interest for an analyst (e.g., sales of
products, treatments of patients, duration of processes, etc.)  are
represented as facts which are associated with cells or points in
this multidimensional space and which are described in terms of a
set of measures.  Thus, every fact is based on a set of dimensions
that determine the granularity adopted for representing the fact's
measures. Dimensions, in turn, are organized as hierarchies of
levels that allow analysts to aggregate data at different levels of
detail.

Hence, two major issues must be faced by designers of MD models:
\begin{enumerate}
\item The adequate representation of interactions between dimensions and
  facts~\cite{DBLP:conf/dmdw/SongRME01}.
\item The adequate representation of relationships between levels of
  aggregation within a dimension
  hierarchy~\cite{DBLP:conf/vldb/JagadishLS99}.
\end{enumerate}

In order to take these issues into consideration, the MD constructs
listed in Tab.~\ref{a1:tab:md-constructs} have to be used. Modeling
by using the full potential of these constructs often results in
complex MD structures. These structures can be designed in a variety
of ways in order to reflect real-world situations, and their
accurate yet understandable design is a cornerstone to enable users
to analyze large amounts of data stored in DWs to effectively and
efficiently support decision-making processes.

\begin{table}
  \centering
  \caption{Constructs for multidimensional modeling}
  \label{a1:tab:md-constructs}
  \begin{tabular}{|l|l|}
    \hline
    Construct & Features\\
    \hline\hline
    Level-Level association & roles, multiplicities, default: many-to-one\\
    Level-Level generalization & default: disjoint and complete\\
    Fact-Dimension association & multiplicities, default: many-to-one\\
    \hline
  \end{tabular}
\end{table}

In order to provide a common vocabulary for MD modeling, the UML
profile proposed in~\cite{DBLP:journals/dke/Lujan-MoraTS06} is used
through this survey.  With this profile, MD models are specified as
UML class diagrams, where facts and dimensions are represented by
\textit{Fact} (\includegraphics[height=3mm]{img/icons/fact}) and
\textit{Dimension}
(\includegraphics[height=3mm]{img/icons/dimension}) classes
respectively. More precisely, \textit{Fact} classes are defined as
composite classes in shared aggregation relationships with several
\textit{Dimension} classes.  If multiplicities are not specified for
those relationships, a default of many-to-one is assumed, i.e., each
fact is associated with one coordinate in every dimension, and each
of the coordinates can be used for many facts. Measures for
\textit{Fact} classes are represented as attributes with the
\textit{FactAttribute} stereotype
(\includegraphics[height=2mm]{img/icons/fa}).  With respect to
dimensions, each level of a dimension hierarchy is specified by a
\textit{Base} class.  Every \textit{Base} class
(\includegraphics[height=3mm]{img/icons/base}) can contain several
dimension attributes (\textit{DimensionAttribute} stereotype,
\includegraphics[height=2mm]{img/icons/da}), and must also contain a
descriptor attribute (\textit{Descriptor} stereotype,
\includegraphics[height=2mm]{img/icons/des}).
 Associations (represented by the stereotype
\textit{Rolls-UpTo}, \includegraphics[height=3mm]{img/icons/roll})
between pairs of \textit{Base} classes form a dimension hierarchy.
On the one hand, roles are used to indicate which level is assumed
to provide a more detailed view than the other: Role \textit{r}, for
roll-up, (resp.~\textit{d}, for drill-down) represents the direction
in which the level of detail decreases (resp.~increases).  On the
other, UML multiplicities are used to specify associations more
precisely.

%Including deg. dimensions and facts???
%This UML profile also allows the definition of degenerate dimensions and
%facts~\cite{book/Kimball/DW}. Degenerate dimensions represent other
%fact features in addition to the measures for analysis. They are
%stereotyped attributes of the \emph{Fact} class
%(\emph{DegenerateDimension} stereotype,
%\includegraphics[height=2mm]{img/dd}). Degenerate facts are used to
%provide further features for every instance combination between
%facts and dimensions. They are represented as an association class
%attached to a relationship between a \textit{Fact} class and a
%\textit{Dimension} class (\emph{DegenerateFact} stereotype,
%\includegraphics[height=3mm]{img/dfact.eps}).


\begin{figure}
  \begin{center}
    \includegraphics[width=\textwidth]{img/dke/example}
  \end{center}
  \caption{Sample scenario} \label{a1:fig:sample}
\end{figure}

For an introductory example consider the conceptual MD model shown
in Fig.~\ref{a1:fig:sample}. Roughly, this model represents a sample
MD scenario, where the facts of interest are sales.  These sales are
structured in a four-dimensional space and allow to analyze who
(dimension \textit{Customer}) bought and who sold (dimension
\textit{Salesperson}), what (dimension \textit{Product}), and when
(dimension \textit{Date}). Concerning multiplicities for
associations between levels we note that the ``standard'' case is
many-to-one ($1..*$ at role \emph{d} and $1$ at $r$) such as in the
case of levels \textit{Day} and \textit{Week}.  Here every day
belongs to exactly one week, and every week consists of several
days. Other cases include the many-to-many relationship ($1..*$ at
both roles) between \textit{Week} and \textit{Month}, which is
commonly referred to as \emph{non-strict}, the many-to-at-most-one
relationship ($1..*$ at role \emph{d} and $0..1$ at \emph{r})
between \textit{Product} and \textit{Category}, which we call
\emph{roll-up incomplete}, as well as the zero-or-more-to-one
relationship ($0..*$ at role \emph{d} and $1$ at \emph{r}) between
\textit{Region} and \textit{Country}, which we call \emph{drill-down
incomplete}. The ``standard'' case for associations between facts
and dimensions is many-to-one ($0..*$ at the role of the \emph{Fact}
class and $1..1$ at the \emph{Dimension} class), for example between
\emph{Sales} and \emph{Date} to indicate that one sale is made at
one certain day, and at a single day several sales can be made.
``Non-standard'' situations include the many-to-many relationship
($*$ at the role of the \emph{Fact} and \emph{Dimension} class)
between \emph{Sales} and \emph{Salesperson}, and the many-to-zero
relationship ($0$ at the role of the \emph{Dimension} class) between
\emph{Sales} and \emph{Customer}.

Importantly, end-user tools for data analysis, such as OLAP (On-Line
Analytical Processing) or ``what-if'' analysis, assume that MD
models ensure summarizability, which refers to the possibility of
accurately computing aggregate values with a coarser level of detail
from values with a finer level of detail. Every MD model to be
implemented must ensure summarizability because, otherwise, its
violation can lead to incorrect results, and therefore erroneous
analysis
decisions~\cite{DBLP:conf/ssdbm/LehnerAW98,DBLP:conf/ssdbm/LenzS97}.
In addition, summarizability is a necessary precondition for
performance optimizations based on
pre-aggregation~\cite{DBLP:conf/vldb/PedersenJD99}.

Therefore, although the sample scenario shown in
Fig.~\ref{a1:fig:sample} takes advantage of highly expressive
constructs to design an understandable conceptual MD model,
summarizability is guaranteed only in case of many-to-one
relationships whereas the other cases require special treatment to
ensure consistent query results.  E.g., consider the reports shown
in Tab.~\ref{a1:tab:DrillDownIncompleteness}. The sales by region
are shown in Tab.~\ref{a1:tab:DrillDownIncompletenessA} (each region
belongs to exactly one country), where the total amount is $60$.
However, if we try to roll-up to country (see
Tab.~\ref{a1:tab:DrillDownIncompletenessB}), the total sales change
to $70$.  Note that as an example for a violation of summarizability
the total numbers obtained in both reports disagree since
``Westfalen'', ``Bayern'' and ``Rheinland'' are german regions,
``Valencia'' and ``Murcia'' are spanish regions, but ``Andorra'' has
no associated regions (as indicated in Fig.~\ref{a1:fig:sample} by
the minimum multiplicity $0$ at role \emph{d} for the association
between region and country) and, therefore, its sales are not
considered in aggregations by region. Importantly, an analyst, who
rolls-up from the report in
Tab.~\ref{a1:tab:DrillDownIncompletenessA} towards the one in in
Tab.~\ref{a1:tab:DrillDownIncompletenessB}, may perceive this
difference as an inconsistency. Further details and examples for
such situations are given in Sect.~\ref{a1:sec:summarizability}.

\begin{table}
\centering \caption{Inconsistent totals for sales due to drill-down
incompleteness}
     \label{a1:tab:DrillDownIncompleteness}
\subtable[Sales by region] {
        \label{a1:tab:DrillDownIncompletenessA}
        \begin{tabular}{|c|c|}
        \hline
        Region & Sales \\
        \hline
        \hline
        Westfalen & 10 \\
        Bayern & 5 \\
        Rheinland & 10 \\
        Valencia & 15 \\
        Murcia & 20 \\
        \hline
        Total & 60 \\
        \hline
        \end{tabular}
} \qquad \qquad \subtable[Sales by country] {
     \label{a1:tab:DrillDownIncompletenessB}
        \begin{tabular}{|c|c|}
        \hline
        Country & Sales \\
        \hline
        \hline
        Germany & 25 \\
        Spain & 35 \\
        Andorra & 10 \\
        \hline
        Total & 70 \\
        \hline
        \end{tabular}
}
\end{table}

Bearing these considerations in mind, the focus of this survey is on
research that addresses how to conceptually model complex MD
structures in an understandable fashion for designers and users,
whilst accurate query responses are allowed by enforcing
summarizability when MD models are being implemented. Although there
is a general agreement concerning basic MD modeling concepts, there
is still a need for unifying terminology, in particular for a
successful treatment of summarizability issues. Therefore, this
survey provides a common vocabulary to integrate existing approaches
for tackling summarizability issues in MD modeling. Our discussion
is structured by dividing the state-of-the-art into two main axes:
\begin{itemize}
\item General approaches for multidimensional modeling. These
approaches focus on defining a structured design process for MD
modeling, giving guidelines to design complex MD structures but
considering summarizability issues only to a certain extent.
\item Multidimensional modeling approaches with special emphasis on complex
structures. These approaches focus on solving summarizability
problems when complex MD structures are defined.
\end{itemize}

This survey is organized as follows.  In the next section we provide
an in-depth description of summarizability concepts and analyze to
what extent complex MD structures affect summarizability.  Later, in
Sect.~\ref{a1:sec:general} and~\ref{a1:sec:specific} we present
current approaches for modeling complex MD structures and sketch how
they tackle summarizability.  We discuss our findings in
Sect.~\ref{a1:sec:discussion} and conclude in
Sect.~\ref{a1:sec:conclusions}.



\section{Summarizability and Multidimensional Modeling}
\label{a1:sec:summarizability} The notion of \emph{summarizability}
was introduced by Rafa\-nelli and
Shoshani~\cite{DBLP:conf/ssdbm/RafanelliS90} in the context of
statistical databases, where it refers to the correct computation of
aggregate values with a coarser level of detail from aggregate
values with a finer level of detail. Although this seminal work on
summarizability is framed within the context of statistical
databases, it is considered as a cornerstone in MD modeling, because
the authors lay the foundations for detecting and avoiding
summarizability problems in an MD space.

Concerning the notion of summarizability, consider a
\emph{Rolls-UpTo} association between two dimension levels, say the
coarser level $l_r$ and the finer level $l_d$, and aggregate values
for $l_d$. According to Rafanelli and
Sho\-sha\-ni~\cite{DBLP:conf/ssdbm/RafanelliS90}, this association
is \emph{summarizable} if ``using'' this association ``yields the
correct summary values'' for $l_r$, and they observe that
many-to-one associations satisfy summarizability while many-to-many
associations violate summarizability.  For example, in the scenario
of Fig.~\ref{a1:fig:sample} if aggregate sales per \textit{Week}
($=l_d$) are given, then there is no way to correctly compute
aggregate sales per \textit{Month} ($=l_r$) using the many-to-many
association between \textit{Week} and \textit{Month} as it is
unknown how the sales of weeks that partially lie in two months need
to be divided.

Furthermore, Rafanelli and Shoshani state four necessary conditions
for summarizability, whose essence translates as follows into our
setting:
\begin{enumerate}
\item Many-to-many associations must not be used, i.e., the maximum
  multiplicity of the coarser level (role \emph{r}) in \emph{Rolls-UpTo} associations
  must be $1$ (instead of $*$).
\item Existing many-to-one associations among levels must be modeled.
\item Many-to-one associations must be ``full,'' i.e., all values contributing
  to the coarser level must be recorded somewhere at the finer level.
  This condition can be enforced by including an additional value ``other'' at
  the finer level, which records potentially missing values.
\item There must not be missing values.
\end{enumerate}
We emphasize that the first two of these conditions deal with the
schema level whereas the remaining two are semantic conditions
concerning the data level.  Indeed, we neglect such completeness
issues in this survey as they embody general data quality problems,
which need to be addressed independently of aggregation: Clearly,
if, e.g., one particular product is missing in the database then no
sales concerning this product are recorded, which in turn implies
that total sales numbers will be incorrect.  Although strictly
speaking summarizability is violated as incorrect aggregate values
are obtained, the core of the problem does not lie in the use of
aggregation.

In the spirit of \cite{DBLP:conf/ssdbm/RafanelliS90}, Lenz and
Shoshani~\cite{DBLP:conf/ssdbm/LenzS97} argue that summarizability
is of most importance for queries concerning multidimensional data,
since violations of this property may lead to erroneous conclusions
and decisions.  Hence, users should be informed when performing
non-summarizable operations.  Importantly, Lenz and Shoshani show
that summarizability is dependent on (i) types of measures and (ii)
the specific dimensions under consideration.  Moreover, they state
three necessary conditions for summarizability.  The first of these
conditions, called \emph{disjointness}, agrees with condition (1) of
Rafanelli and Shoshani stated above.  The second condition, called
\emph{completeness}, includes condition (4) above and requires in
addition that the minimum multiplicity at both ends of an
association is $1$ (instead of $0$).  The third condition, called
\emph{type compatibility}, ensures that the aggregate function
applied to a measure is summarizable according to the type of the
measure (stock, flow and value-per-unit) and the type of the related
dimensions (temporal, non-temporal).  E.g., account balances or
quantity on hand are of type stock and, hence, must not be summed
over the time dimension.  (In contrast, computing average balances
over time or sums of balances over products is reasonable.)

For the special case where aggregate functions are restricted to the
\emph{sum} operator, the term \emph{additivity} (instead of
summarizability) is used frequently. An in-depth analysis of
additivity and a taxonomy for reasons why additivity may not hold is
presented by Horner and Song~\cite{DBLP:conf/er/HornerS05}.  They
distinguish schema problems, which are our focus in this survey,
from data problems (e.g., inconsistencies and imprecision) and
computational problems (e.g., type compatibility in the sense of
\cite{DBLP:conf/ssdbm/LenzS97}), give typical examples for each
problematic case, and suggest guidelines for their management.
Concerning schema problems, in particular roll-up and drill-down
incomplete as well as non-strict associations, the general ideas are
(i) to tolerate and display incorrect results and (ii) to measure
and display inaccuracies, which enables analysts to estimate whether
and how much results may suffer from summarizability problems.

Finally, although Lenz and Shoshani~\cite{DBLP:conf/ssdbm/LenzS97}
only focus on the relationships between two levels of a dimension
hierarchy, the relationships between facts and dimensions can also
cause summarizability problems in MD modeling. Specifically, to
avoid erroneous results when an MD model is queried, every measure
in the fact must be determined by all dimensions, (i.e., the maximum
and minimum multiplicity at the dimension end must be $1$ in the
relationship between a fact and a dimension), which we also address
in the following.

\subsection{Constraints for Summarizability}
\label{a1:sec:dependencies}

It is instructive to note that the conditions for summarizability
given in~\cite{DBLP:conf/ssdbm/RafanelliS90} and
in~\cite{DBLP:conf/ssdbm/LenzS97} are incomparable: Condition (3)
of~\cite{DBLP:conf/ssdbm/RafanelliS90} is not covered
by~\cite{DBLP:conf/ssdbm/LenzS97}, and condition (3)
of~\cite{DBLP:conf/ssdbm/LenzS97} is not covered
by~\cite{DBLP:conf/ssdbm/RafanelliS90}.  Nevertheless, the
conditions related to disjointness, completeness, and presence of
many-to-one associations can all be understood as constraints (or
dependencies) that are expressible at the schema level (e.g., in
terms of multiplicities as done above).  In fact, several authors
advocate to represent summarizability conditions explicitly via
dependencies at the schema level.

Importantly, Lehner et al.~\cite{DBLP:conf/ssdbm/LehnerAW98}
realized that many-to-one associations between dimension level are
just functional dependencies (FDs) as known from standard relational
database theory, and they distinguish strong FDs, which correspond
to total functions, from weak ones, which correspond to partial
functions and which give rise to roll-up incomplete associations in
our terminology.

Similarly, Niemi et
al.~\cite{DBLP:conf/dmdw/NiemiNT01,DBLP:journals/dke/NiemiNT03} also
observe that FDs in dimension hierarchies avoid summarizability
problems. Moreover, they mention the use of Boolean dependencies to
deal with special cases of many-to-many associations, and they
propose novel dependencies to avoid roll-up and
drill-incompleteness.

Based on the work of~\cite{DBLP:conf/ssdbm/LehnerAW98},
Lechtenb{\"o}rger and
Vo\-ssen~\cite{DBLP:journals/is/LechtenborgerV03} define three
multidimensional normal forms (MNFs).  Intuitively, the first one
(1MNF) deals with the adequate representation of an MD model based
on the FDs that hold in the underlying data sources.  Importantly,
1MNF implies the first two conditions for summarizability of
Rafanelli and Shoshani~\cite{DBLP:conf/ssdbm/RafanelliS90}.
Furthermore, the second and third MNF allow to model accurately when
and why summarizability may not be given according to the conditions
of~\cite{DBLP:conf/ssdbm/LenzS97,DBLP:conf/ssdbm/RafanelliS90} but
can be ensured in a context-sensitive manner based on schema
information.  For example, the specialization of customers into
private customers and companies shown in Fig.~\ref{a1:fig:sample}
cannot be represented directly in most MD modeling approaches.
Instead, those approaches would include roll-up incomplete
associations from \textit{Customer} to \textit{Job} as well as from
\textit{Customer} to \textit{Branch}.  Now, to enrich modeling
approaches without explicit presence of specialization constructs,
the work~\cite{DBLP:journals/is/LechtenborgerV03} introduces
\emph{context dependencies}, which enable an implicit representation
of such specializations (and their reconstruction in relational
implementations).

A context dependency of~\cite{DBLP:journals/is/LechtenborgerV03} can
be regarded as a restricted kind of \emph{dimension constraint} in
the sense of~\cite{DBLP:journals/tods/HurtadoGM05}.  Hurtado et
al.~\cite{DBLP:journals/tods/HurtadoGM05} point out that there are
two kinds of dimension hierarchies: homogeneous and heterogeneous.
The former fulfills the summarizability conditions, whereas the
latter does not. While heterogeneous dimensions have hierarchies
where two members in a given level are allowed to have ancestors in
different levels, homogeneous dimensions contain the typical
normalized hierarchies without summarizability problems. The authors
argue that it is easier for designers to model heterogeneous
dimensions because they are closer to real-world (they represent
more naturally and cleanly many practical situations).  In our
context, heterogeneous dimensions correspond to roll-up/drill-down
incomplete hierarchies (since the mapping between levels in
heterogeneous dimensions is defined as partial), whereas non-strict
associations are not addressed in
\cite{DBLP:journals/tods/HurtadoGM05}.  The aim of Hurtado et al.\
is to reason about summarizability of heterogeneous dimensions via a
new kind of integrity constraints, called \emph{dimension
constraints}, for which they derive a summarizability test.
Moreover, they introduce the notion of frozen dimensions, which
represent minimal homogeneous dimensions mixed up in a heterogeneous
dimension, and they provide an algorithm for the implication problem
of dimension constraints based on frozen dimensions.

The constraints for summarizability presented so far aim at a
careful definition of dimensions and their hierarchies.  Hence, they
are called intra-dimensional constraints
in~\cite{DBLP:conf/ssdbm/LehnerAW98}.  However, within the full
scope of MD modeling, further inter-dimensional
constraints~\cite{DBLP:conf/ssdbm/LehnerAW98} for relationships
between facts and dimensions are needed to ensure summarizability.
These interdimensional constraints are related to the grain of the
fact in such a way that, to avoid erroneous results when an MD model
is queried, every measure in the fact must be determined by all
dimensions, which is reflected in the common relational
implementation of a star schema, where the primary key of the fact
table is composed of foreign keys of the dimension
tables~\cite{book/Kimball/DW}. These constraints are made formally
precise by the 1MNF proposed
in~\cite{DBLP:journals/is/LechtenborgerV03}. Indeed, 1MNF makes sure
that the terminal dimension levels of all dimensions of a fact form
a key for every measure and implies that all measures are recorded
with the same granularity.

In order to obtain a complete picture of the problem space, which
allows to pinpoint and differentiate previous approaches, we next
revisit each of the UML constructs listed in
Tab.~\ref{a1:tab:md-constructs} that are used to model complex MD
structures.  For each construct, we systematically discuss the
arising summarizability issues based on a complete enumeration of
cases.

\subsection{Relationships between Dimension Levels}
Dimension hierarchies are among the most important MD structures to
be modeled, since they are used by data analysis tools to accurately
aggregate or disaggregate data, depending on levels of aggregation.
These levels of aggregation must be explicitly specified by
organizing the members of a given dimension into hierarchies of
levels, in particular in the presence of various kinds of
``irregularity'' or ``heterogeneity'' for which early examples can
be found in~\cite{DBLP:conf/ssdbm/LehnerAW98}.

Importantly, all kinds of relationships between pairs of dimension
levels that have been proposed so far can be represented either by
associations or by generalizations of dimension levels as listed in
Tab.~\ref{a1:tab:md-constructs}. Concerning associations, we have
already seen that multiplicities play a crucial role for
summarizability.  In view of that observation in
Tab.~\ref{a1:tab:multiplicity-classification} we present a complete
characterization of associations based on the minimum and maximum
multiplicities used in the roles \emph{d} and \emph{r}.  In that
table ``regular'' and ``unusual'' denote association types without
summarizability problems, the latter being rarely used, whereas the
remaining entries form a selection of terms used in the literature
for a particular irregularity; the ones used in this paper are
\emph{emphasized} and explained in the following.  In particular, we
propose the novel terms ``drill-down incomplete'' and ``roll-up
incomplete,'' which convey a figurative meaning that we hope to be
easy to remember.

Moreover, we emphasize that our discussion deals with single
associations in contrast to entire hierarchies, which allows for a
more precise classification and treatment of summarizability issues
(in particular, a single hierarchy may contain unproblematic
associations as well as problematic associations of different
kinds).

Concerning generalization we briefly observe that disjointness of
generalizations bears similarity with strictness, whereas
completeness corresponds to roll-up completeness.  Details are
presented below.

\begin{table}
    %\tiny
  \centering
  \caption{Classification of associations between dimension levels}
  \label{a1:tab:multiplicity-classification}
    \begin{tabular}{|c||c|c|c|c|}
         \hline
       & \multicolumn{2}{|c|}{Minimum Multiplicity} & \multicolumn{2}{|c|}{Maximum Multiplicity} \\\hline
       & 0 & 1 & 1 & * \\\hline
      Role d & \emph{drill-down incomplete}, asymmetric, non-onto, unbalanced & regular & unusual & regular \\
      Role r & \emph{roll-up incomplete}, incomplete, non-covering, ragged & regular & regular & \emph{non-strict} \\\hline
    \end{tabular}
\end{table}

\subsubsection{Regular Relationships between Dimension Levels} We
note that summarizability of the ``regular'' entries follows from
the necessary conditions ``disjointness'' and ``completeness'' for
summarizability stated in \cite{DBLP:conf/ssdbm/LenzS97}. Indeed,
disjointness implies that the maximum multiplicity at role \emph{r}
is $1$ while completeness implies that the minimum multiplicities at
both roles are $1$.  Furthermore, if the maximum multiplicity at
role \emph{d} is $1$ an unusual situation arises; however, this
situation does not contradict the necessary conditions of
\cite{DBLP:conf/ssdbm/LenzS97}.

\subsubsection{Drill-down Completeness}
A \emph{Rolls-UpTo} association involving a pair of dimension levels
is \emph{drill-down complete} if for every element $e$ of the
coarser level (i.e., role r, such as \textit{Country} for the
association between \textit{City} and \textit{Country}) there exists
an element at the finer level (i.e., role d, here \textit{City})
which is associated with element $e$; otherwise, it is called
\emph{drill-down incomplete}.  In other words, a \emph{Rolls-UpTo}
association is drill-down incomplete if the minimum multiplicity at
role \emph{d} is $0$; otherwise, it is drill-down complete.  For
example, the association between \textit{Country} and
\textit{Region} in Fig.~\ref{a1:fig:sample} is drill-down incomplete
as there are countries (such as Andorra, Monaco, etc.)  without
associated regions.  As explained in the Introduction and
illustrated in Tab.~\ref{a1:tab:DrillDownIncompleteness}, drill-down
incompleteness violates summarizability since it may yield
inconsistent totals.

\subsubsection{Roll-up Completeness}
A \emph{Rolls-UpTo} association involving a pair of dimension levels
is \emph{roll-up complete} if for every element $e$ of the finer
level (i.e., role d, such as \textit{Product} for the association
between \textit{Product} and \textit{Brand}) there exists an element
at the coarser level (i.e., role r, here \textit{Brand}) which is
associated with element $e$; otherwise, it is called \emph{roll-up
incomplete}.  In other words, a \emph{Rolls-UpTo} association is
roll-up incomplete if the minimum multiplicity at role \emph{r} is
$0$; otherwise, it is roll-up complete.  For example, the
association between \textit{Product} and \textit{Category} in
Fig.~\ref{a1:fig:sample} is roll-up incomplete, and faces the
problem of inconsistent totals as shown in
Tab.~\ref{a1:tab:RollUpIncompleteness}, where we assume that
``milk'' and ``beer'' belong to category ``drink'', ``bread'' and
``tuna'' to category ``food'', and ``napkin'' has no category.
Therefore, when factual data is aggregated by product the sales are
$60$ (see Tab.~\ref{a1:tab:RollUpIncompletenessA}). However, special
attention should be paid when data is aggregated by category, since
``napkin'' sales are not taken into account and the total sales
decrease to $40$ (as shown in
Tab.~\ref{a1:tab:RollUpIncompletenessB}).

\begin{table}
\centering \caption{Inconsistent totals for sales due to roll-up
incompleteness}
     \label{a1:tab:RollUpIncompleteness}
\subtable[By product] {
        \label{a1:tab:RollUpIncompletenessA}
        \begin{tabular}{|c|c|}
        \hline
        Product & Sales \\
        \hline
        \hline
        Milk & 10 \\
        Beer & 5 \\
        Bread & 10 \\
        Tuna & 15 \\
        Napkin & 20 \\
        \hline
        Total & 60 \\
        \hline
        \end{tabular}
} \qquad\qquad \subtable[By category] {
     \label{a1:tab:RollUpIncompletenessB}
        \begin{tabular}{|c|c|}
        \hline
        Category & Sales \\
        \hline
        \hline
        Drink & 15 \\
        Food & 25 \\
        \hline
        Total & 40 \\
        \hline
        \end{tabular}
}
\end{table}


\subsubsection{Strictness}
A \emph{Rolls-UpTo} association involving a pair of dimension levels
is \emph{strict} if for every element $e$ of the finer level (i.e.,
role d, such as \textit{Day} for the association between
\textit{Day} and \textit{Week}) there exists at most one element at
the coarser level (i.e., role r, here \textit{Week}) which is
associated with element $e$; otherwise, it is called
\emph{non-strict}.  In other words, a \emph{Rolls-UpTo} association
is strict if the maximum multiplicity at role \emph{r} is $1$;
otherwise, it is non-strict.  For example, the association between
\textit{Week} and \textit{Month} in Fig.~\ref{a1:fig:sample} is
non-strict, and requires special care to avoid the well-known double
counting problem, which is illustrated in
Tab.~\ref{a1:tab:NonStrictness}: As week ``5-2008'' partially
belongs to ``January'' as well as ``February'' (see
Tab.~\ref{a1:tab:NonStrictnessA}), the sales for week ``5-2008''
should not be counted twice (as is done in
Tab.~\ref{a1:tab:NonStrictnessB}) but should be divided
appropriately among both months.

\begin{table}
\centering \caption{Double counting problem for sales due to
non-strictness}
     \label{a1:tab:NonStrictness}
\subtable[Sales by week] {
        \label{a1:tab:NonStrictnessA}
        \begin{tabular}{|c|c|}
        \hline
        Week & Sales \\
        \hline
        \hline
        4-2008 & 10 \\
        5-2008 & 20 \\
        6-2008 & 10 \\
        7-2008 & 10 \\
        8-2008 & 10 \\
        9-2008 & 10 \\
        \hline
        Total & 70 \\
        \hline
        \end{tabular}
} \qquad\qquad \subtable[Sales by month] {
     \label{a1:tab:NonStrictnessB}
        \begin{tabular}{|c|c|}
        \hline
        Month & Sales \\
        \hline
        \hline
        Jan-2008 & 30 \\
        Feb-2008 & 60 \\
        \hline
        Total & 90 \\
        \hline
        \end{tabular}}
\end{table}


\subsubsection{Generalization}
As observed by Lehner et al.~\cite{DBLP:conf/ssdbm/LehnerAW98},
dimension levels in MD models may exhibit heterogeneity in the sense
that certain properties may only be applicable to a subset of
elements. E.g., for the dimension level \textit{Customer} in the
scenario shown in Fig.~\ref{a1:fig:sample}, some customers can be
categorized by their \textit{Gender} (if they are human beings) and
others by \textit{ContactPerson} (if they are companies).  If those
properties are modeled as levels, which are reachable from
\textit{Customer} via \emph{Rolls-UpTo} associations, then those
associations will necessarily suffer from roll-up incompleteness and
the inherent summarizability problems. Moreover, as explained
in~\cite{DBLP:conf/ssdbm/LehnerAW98}, sparse data cubes may result,
and users may pose inconsistent queries (``show sales by Gender and
ContactPerson usage'').

From an object-oriented perspective such heterogeneity indicates the
existence of sub-classes where the individual properties are either
applicable (as attributes) or not.  In fact, this perspective guides
the definition of the third multidimensional normal
of~\cite{DBLP:journals/is/LechtenborgerV03}, where context
dependencies explain the applicability of dimension levels;
moreover, these dependencies can be used to construct class
hierarchies that contain the applicable attributes, which avoids
roll-up incompleteness and associated summarizability problems.  It
is instructive to note, however, that this approach does \emph{not}
allow generalizations as modeling constructs.  Moreover, although
several MD design proposals are based on object-oriented modeling
and, in particular, the
UML~\cite{DBLP:journals/is/AbelloSS06,DBLP:journals/dke/Lujan-MoraTS06,DBLP:journals/dss/PratAC06}
none of them explicitly suggests the use of generalization to avoid
roll-up incompleteness.  In line with~\cite{Bodart01} we argue that
optional constructs such multiplicities of $0$ should be avoided in
\emph{understandable} conceptual models whenever possible.  In
particular, generalization relationships for dimension levels embody
an attractive alternative for roll-up incomplete associations.

With generalization, new analysis potential is unleashed.  As
explained in~\cite{DBLP:journals/is/LechtenborgerV03}, (i)
traditional roll-up operations may be performed within a certain
sub-class by using only the applicable levels, which now provides
context-sensitive summarizability, or (ii) a novel type of
drill-down operation allows to switch from \emph{Base} classes to
their immediate sub-classes, e.g., in Fig.~\ref{a1:fig:sample} sales
for all \textit{Customer}s are split into sales for \textit{Company}
and \textit{Private} customers (or in the opposite direction via a
novel type of roll-up).  In traditional MD modeling approaches case
(ii) can be simulated by introducing a new dimension level that
captures the immediate sub-classes and that is connected from the
base class via a \emph{Rolls-UpTo} association (e.g., there is a
\emph{Rolls-UpTo} association from \textit{Customer} --role
\emph{d}-- to the new level \textit{CustomerType} --role \emph{r}--
with elements ``company'' and ``private'').  It is easy to see that
this association has
\begin{itemize}
\item multiplicity $1$ at role \emph{r} if the generalization is complete and
  disjoint,
\item multiplicity $0..1$ at role \emph{r} if the generalization is incomplete
  and disjoint,
\item multiplicity $1..*$ at role \emph{r} if the generalization is complete
  and overlapping, and
\item multiplicity $0..*$ at role \emph{r} if the generalization is incomplete
  and overlapping.
\end{itemize}
Consequently, the novel type of drill-down and roll-up operations is
summarizable for complete and disjoint generalizations, whereas
other cases exhibit the summarizability issues of roll-up incomplete
and/or non-strict associations seen above.


\subsection{Fact-Dimension Relationships}
A crucial decision for designing multidimensional models concerns
the grain of facts~\cite{book/Kimball/DW}, i.e., the list of
dimensions which defines the scope of the measures in a fact.
Therefore, the grain of the fact is determined by fact-dimension
relationships. To avoid erroneous results, an MD model must have a
consistent granularity, which means that every measure in the fact
must be determined by all dimensions. As we have explained in
Sect.~\ref{a1:sec:dependencies} this assumption is made formally
precise by the first multidimensional normal form proposed
in~\cite{DBLP:journals/is/LechtenborgerV03}. This assumption implies
that the relationship between a fact and a dimension must be
many-to-one, which avoids summarizability problems. Therefore,
multidimensional models are usually defined according to this
multiplicity constraint in order to enforce summarizability in
fact-dimension relationships. However, these constraints are too
strict for modeling many real-world situations in an easy and
understandable way. Designers must deal with scenarios in which
different granularities are necessary and where relationships
between a fact and a dimension can have different multiplicities.
Such situations can lead to MD models with summarizability problems.
Specifically, Tab.~\ref{a1:tab:FD-multiplicity-classification}
presents different kinds of fact-dimension relationships, which are
characterized by the multiplicities of the association between fact
and dimension.

\begin{table}
    %\tiny
  \centering
  \caption{Classification of associations between facts and dimensions}
  \label{a1:tab:FD-multiplicity-classification}
    \begin{tabular}{|c||c|c|c|c|}
         \hline
       & \multicolumn{2}{|c|}{Minimum Multiplicity} & \multicolumn{2}{|c|}{Maximum Multiplicity} \\\hline
       & 0 & 1 & 1 & * \\\hline
      Fact & regular & unusual & unusual & regular \\
      Dimension & \emph{incomplete dimensioning} & regular & regular & \emph{non-strict dimensioning}, many-to-many \\\hline
    \end{tabular}
\end{table}

Again, ``regular'' and ``unusual'' means that summarizability is
ensured, whilst the other terms denote situations which violate
summarizability.

\subsubsection{Regular Fact-Dimension Relationships} Summarizability is ensured in these relationships, because the fact
is functionally determined by the dimension. The common feature of
these regular situations is that the minimum and maximum
multiplicity in the role of the \emph{Dimension} class is $1$ to
indicate that every instance of the \emph{Fact} class must be always
related to one and only one instance of the \emph{Dimension} class.
The other multiplicities are described as follows:

\begin{itemize}

\item Minimum multiplicity of $0$ at the end of the fact. This
multiplicity allows the existence of dimension instances that are
not related with any fact instance. This is the most common option
for multidimensional modeling, e.g., consider a \emph{Product}
dimension where some products have not been sold so far.

\item Minimum multiplicity of $1$ at the end of the fact. This
multiplicity requires that every dimension instance is related with
at least one fact instance. For practical purposes this multiplicity
is usually ignored, since it introduces additional restrictions in
the multidimensional model that make ETL
(Extrac\-tion-Transformation-Load) processes more complex and prone
to fail.

\item Maximum multiplicity of $1$ at the end of the fact. This
multiplicity requires that every dimension instance is related with
at most one fact instance. For practical purposes this multiplicity
is usually ignored as well, since it prevents the orthogonal use of
dimensions. E.g., consider a \emph{Time} dimension where every date
can only be used once.

\item Maximum multiplicity of $*$ at the end of the fact. This
multiplicity allows dimension instances to be related with many fact
instances. It is the most desirable option within the regular
situations.
\end{itemize}


\subsubsection{Incomplete Fact-Dimension Relationships} An
association between a fact $f$ and a dimension $d$ is complete if
for every fact instance of $f$, there exists a dimension instance of
$d$ which is related to that fact instance; otherwise, the
association is \emph{incomplete}. This situation can fall into a
summarizability violation since there is a granularity mismatch in
the instances of the fact. In other words, a fact-dimension
association is incomplete if the minimum multiplicity at the end of
the dimension is $0$. For example, the association between
\emph{Customer} dimension and \emph{Sales} fact in
Fig.~\ref{a1:fig:sample} exhibits an incomplete relationship. This
sample faces the problem of inconsistent totals, as shown in
Tab.~\ref{a1:tab:DimensioningIncompletenessA}, where we assume that
John and Anna buy some products in January, and George goes shopping
in April. The totals arise in a supermarket where some customers
have loyalty cards to get discounts. For those customers, sales are
recorded directly together with their personal information (e.g.,
city of residence). In contrast, sales of (anonymous) customers
without cards are recorded without considering any personal
information (see Tab.~\ref{a1:tab:DimensioningIncompletenessB}).
Consequently, when the sales are analyzed by customer and date some
sales are missing (those from anonymous customers). Only when the
customer is not taken into account, the total sales are correct. The
problem of inconsistency is shown in
Tab.~\ref{a1:tab:DimensioningIncompleteness} where some anonymous
sales made in January are not shown when customer dimension is
considered.

\begin{table}
\centering \caption{Inconsistent totals for sales due to
incompleteness}
     \label{a1:tab:DimensioningIncompleteness}
\subtable[Sales by customer and time] {
        \label{a1:tab:DimensioningIncompletenessA}
        \begin{tabular}[t]{|c|c|c|}
        \hline
        Customer & Date & Sales \\
        \hline
        \hline
        John & January-2001 & 10 \\
        Anna & January-2001 & 5 \\
        George & April-2001 & 15 \\
        \hline
       \multicolumn{2}{|c|}{Total} & 30 \\
        \hline
        \end{tabular}
} \qquad\qquad \subtable[Sales by time] {
     \label{a1:tab:DimensioningIncompletenessB}
        \begin{tabular}[t]{|c|c|}
        \hline
        Date & Sales \\
        \hline
        \hline
        January-2001 & 25 \\
        April-2001 & 15 \\
        \hline
        Total & 40 \\
        \hline
        \end{tabular}}
\end{table}


\subsubsection{Non-strict Fact-Dimension Relationships} An association
between a fact $f$ and a dimension $d$ is strict if for every
instance of the fact $f$ there exists at most one instance of the
dimension $d$ which is related to that fact instance; otherwise, it
is called \emph{non-strict}. In other words, non-strict associations
between a fact and a specific dimension are specified by means of
the multiplicity $*$ in the role of the corresponding dimension.
This situation may cause summarizability problems in the same way as
the double counting problem for non-strict dimension associations
explained above. For example, in Fig.~\ref{a1:fig:sample}, the
association between the \emph{Sales} fact and the \emph{Salesperson}
dimension is non-strict, which means that more than one salesperson
could be involved in the same sale. This problem is illustrated in
Tab.~\ref{a1:tab:DimensioningNonStrictness}: as the sale made on
``January 17th'' is shared by ``Bill'' and ``Peter'', it should be
counted once.

%Jose: Including degenerate facts or weighting factor?
%Within this context, degenerate facts are also important in order to
%indicate further features for every instance combination in a
%many-to-many relationship with a particular
%dimension~\cite{book/Kimball/DW}.

\begin{table}
\centering \caption{Double counting problem for sales due to
non-strictness}
     \label{a1:tab:DimensioningNonStrictness}
\subtable[Sales by salesperson] {
     \label{a1:tab:DimensioningNonStrictnessA}
        \begin{tabular}[t]{|c|c|c|}
        \hline
        Date & Salesperson & Sales \\
        \hline
        \hline
        17/01/2001 & Bill & 10 \\
        17/01/2001 & Peter & 10 \\
        18/01/2001 & Bill & 5 \\
        18/01/2001 & Peter & 5 \\
        \hline
        \multicolumn{2}{|c|}{Total} & 30 \\
        \hline
        \end{tabular}
} \qquad \subtable[Shared sales by salesperson] {
     \label{a1:tab:DimensioningNonStrictnessB}
        \begin{tabular}[t]{|c|c|c|}
        \hline
        Date & Salesperson & Sales \\
        \hline
        \hline
        17/01/2001 & Bill, Peter & 10 \\
        18/01/2001 & Bill & 5 \\
        18/01/2001 &  Peter & 5 \\
        \hline
        \multicolumn{2}{|c|}{Total} & 20 \\
        \hline
        \end{tabular}}
\end{table}

\subsubsection{Beyond Multiplicities} In addition to summarizability issues arising from problematic multiplicities,
there is another more subtle way in which an inadequate choice for
the grain of facts can lead to summarizability problems.  As an
example, consider the scenario shown in Fig.~\ref{a1:fig:sample} and
assume that there is an additional measure recording the age of
customers.  If this measure is included in the fact \textit{Sales},
then duplicate age values will be recorded for every product bought
by a customer within the same year.  Formally, these duplicates
arise as the age is functionally dependent on only two of the three
dimensions, namely on the levels \textit{Date} and
\textit{Customer}.  Indeed, such a scenario violates the first
multidimensional normal form defined
in~\cite{DBLP:journals/is/LechtenborgerV03} (see also
Sect.~\ref{a1:sec:dependencies}).  Notice that while in this example
it may seem obvious \emph{not} to record customer ages for customer
sales, in more complex scenarios such situations may be more
difficult to detect.  In this survey, we assume that MD models
satisfy the first multidimensional normal form, which can be
guaranteed based on an analysis of functional dependencies occurring
in the application domain, for example using the tool Data Warehouse
Detective~\cite{DBLP:conf/btw/HaselmannLV07}.


\section{General Approaches for Multidimensional Modeling}
\label{a1:sec:general} When implementing the MD model in a
relational platform, the most common representation is the star
sche\-ma~\cite{book/Kimball/DW}. This schema consists of one central
fact table and several related dimension tables. The levels of
aggregation are implicitly contained within the dimension tables. As
the star schema is a \emph{logical} representation that aims at a
relational implementation, it does not consider the modeling of
complex MD structures and it only ensures summarizability if the
relationships between MD elements are many-to-one. Therefore,
several \emph{conceptual} modeling approaches have been proposed to
design MD models that address the design of complex MD structures to
accurately model the real world.

In this section, we first review early \emph{conceptual} modeling
approaches such as~\cite{DBLP:journals/ijcis/GolfarelliMR98}
and~\cite{DBLP:conf/dmdw/HusemannLV00}, which focus on a structured
design process starting from requirements and ending in a
tool-specific implementation but which lack support for defining
complex MD structures. Afterwards, we consider the recent
approaches~\cite{DBLP:journals/is/AbelloSS06,DBLP:journals/dke/Lujan-MoraTS06,DBLP:journals/dss/PratAC06},
which exhibit more expressiveness based on an object-oriented
perspective but which lack in a detailed proposal for ensuring
summarizability.

In~\cite{DBLP:journals/ijcis/GolfarelliMR98}, the authors present a
graphical conceptual model for designing MD models (Dimensional Fact
Mo\-del). This conceptual model consists of a set of fact schemata
whose basic elements are facts, measures, attributes, dimensions,
and hierarchies. The relationship between a fact and a specific
dimension is always many-to-one, and only strict hierarchies can be
explicitly modeled by means of many-to-one links between dimension
attributes to explicitly indicate how to aggregate and disaggregate
data. Furthermore, the existence of optional relationships between
attributes in a hierarchy and type compatibility (the third
condition for summarizability of~\cite{DBLP:conf/ssdbm/LenzS97}) of
fact attributes along dimensions is explicitly considered. This
approach does not deal with obtaining an implementation of the
conceptual MD model without summarizability problems because it only
uses constructs that do not violate summarizability conditions.

The MD model proposed in~\cite{DBLP:conf/dmdw/HusemannLV00} stresses
the design of summarizable dimension hierarchies in the presence of
optional dimension levels and advocates the specification of
summarizability constraints.  The authors propose a graphical
notation for conceptual MD design and also show how to apply
multidimensional normal
forms~\cite{DBLP:conf/ssdbm/LehnerAW98,DBLP:conf/dmdw/HusemannLV00,DBLP:journals/is/LechtenborgerV03}
in order to guarantee summarizability.  Dimension hierarchies are
classified into two basic types: (i) a simple hierarchy consists of
exactly one aggregation path within a dimension, and (ii) a multiple
dimension hierarchy contains at least two different aggregation
paths in a dimension.  Alternative aggregation paths occur in
multiple hierarchies if elements of split levels belong to exactly
one element of a higher level.  Moreover, optional groups of
aggregation paths are also allowed if some element of a dimension
level does not belong to an element of some higher level. Therefore,
this approach considers not only strict hierarchies but also
generalization relationships between dimension levels. Regarding the
relationships between facts and dimensions, only many-to-one
relations are considered, and multidimensional normal forms are used
in this approach to ensure that the measures are determined by the
set of dimensions.

Concerning semantically richer proposals,
in~\cite{DBLP:journals/dss/PratAC06}, Prat et al. first build a
general UML (Unified Modeling Language) diagram which is enriched
with MD concepts that facilitate the derivation of a logical
representation. Although, the authors describe in another
paper~\cite{DBLP:conf/er/AkokaCP01} how to model different kinds of
dimension hierarchies, the transformations of the hierarchies into a
logical level are not as rich, since they are only defined from
many-to-one and one-to-one associations of the conceptual model to
fulfill summarizability conditions. Moreover, many-to-many
relationships in the general UML diagram are mapped to facts
directly, whereas roll-up/drill-down incompleteness is not
addressed. Type compatibility is considered in this work, since the
link between a measure and a dimension is characterized by a set of
aggregate functions depending on the type of the measure and the
corresponding dimension.

Another approach which uses UML constructs is described
in~\cite{DBLP:journals/is/AbelloSS06}, where the authors define the
MD data model YAM$^{\mbox{2}}$.  Here, a dimension is a connected,
directed graph, where every vertex corresponds to a hierarchy level,
and an edge reflects that every instance of target level can be
decomposed into a collection of instances of the source level (a
relation between levels reflects a part-whole relationship among
instances of levels). Although rich constructs are provided,
hierarchies must conform with the three conditions for
summarizability of \cite{DBLP:conf/ssdbm/LenzS97} (disjointness,
completeness and type compatibility) at the conceptual level. The
authors argue that the first two conditions depend on constraints
specified over multiplicities of relationships between levels, and
they propose to allow different kinds of hierarchies provided that
multiplicity information is taken into account to decide whether
summarizability is given (allowing non-strict hierarchies). However,
neither roll-up incomplete nor drill-down incomplete hierarchies are
allowed because every instance of a dimension must have the same
structure; to address this shortcoming, the authors propose a
solution at the instance level based on dummy values. Concerning the
third condition, YAM$^{\mbox{2}}$ allows to specify information
related to type compatibility directly in the metamodel. Moreover,
YAM$^{\mbox{2}}$ covers multiple hierarchies in each dimension as
well as generalization relationships between levels. Modeling
non-strict relationships between facts and dimensions is considered
by means of multiplicities in the UML association, whilst
incompleteness is addressed by defining generalization between facts
in order to express optionality. Unfortunately, the approach does
not provide mechanisms to avoid summarizability problems that arise
during schema implementation.

Finally, in~\cite{DBLP:journals/dke/Lujan-MoraTS06}, a UML profile
for MD modeling is proposed.  In this approach, a dimension is
composed of hierarchy levels.  An association between levels
specifies the relationship between two levels of a hierarchy.  The
only prerequisite is that these levels must define a Directed
Acyclic Graph (DAG) rooted in the dimension. A dimension contains a
unique first hierarchy level called terminal dimension level.  An
aggregation path is a subsequence of hierarchy levels, which starts
in a terminal level (lower level of detail).  The definition of a
dimension hierarchy is very expressive, since there are no
restrictions concerning associations between dimension levels,
provided that every hierarchy fulfills the DAG condition.
Consequently, every kind of relationship between levels of a
dimension can be represented by using the corresponding multiplicity
in the association between levels. Furthermore, it is worth
mentioning that this work considers the definition of non-strict and
incomplete relationships between facts and dimensions via the
definition of different multiplicities between facts and dimensions.
However, this work neither offers guidelines to help the designer
model different kinds of complex MD structures nor to ensure
summarizability.  Concerning type compatibility, all measures are
considered as additive by default, i.e., measure values can be
summed along all dimensions. Non-additivity and semi-additivity are
considered by defining constraints on measures between brackets and
placing them somewhere around the fact. These constraints have
formal underlying formulae and contain the allowed operators, if
any, along the dimension that the measure is not additive.  Finally,
in~\cite{journals/dss/Mazon2008,journals/dke/Mazon2007}, this
approach is integrated in a model driven framework in order to
obtain a logical representation of the conceptual MD model in an
automatic way.

\section{Multidimensional Modeling of Complex Structures}
\label{a1:sec:specific} In this section, we review approaches that
improve the proposals previously mentioned via the definition of
mechanisms to facilitate the modeling of complex MD structures and
their implementation ensuring summarizability. These approaches are
concerned with either (i) how to model complex dimension hierarchies
or (ii) how to model relationships between facts and dimensions.

\subsection{Dimension Hierarchies}
\label{a1:sec:dim-hierarchies} One fundamental work has been carried
out by Pedersen et
al.~\cite{DBLP:conf/vldb/PedersenJD99,DBLP:journals/is/PedersenJD01},
in which the authors argue that summarizability occurs when
dimension hierarchies are ``normalized,'' i.e., roll-up and
drill-down complete as well as strict.  Importantly, starting from
an MD data model allowing multiple, drill-down and roll-up
incomplete, as well as non-strict hierarchies, the authors show how
to (i) transform dimension \emph{instances} to enforce
summarizability and (ii) implement transformed hierarchies using
relational database technology. The authors argue that an MD
modeling approach should support the explicit design of every kind
of hierarchy at a conceptual level to model real-world scenarios
accurately and at the same time easily. Only later at a logical
phase, summarizability constraints must be enforced by transforming
hierarchies into well-behaved logical structures that enable
summarizability when data analysis tools are used. For doing so,
instance level algorithms are presented to automatically transform
dimension hierarchies to achieve summarizability for hierarchies
that are roll-up/drill-down incomplete or non-strict. As this
proposal works at the instance level, it is necessary to transform
the data that will populate the DW, which may involve considerable
efforts of preprocessing. In particular, ETL processes become more
complex, as summarizability checks must be incorporated and executed
for every update.  In addition, as the data transformations produce
artificial data values, data analysis becomes more complex.

In~\cite{DBLP:conf/caise/MalinowskiZ04,DBLP:journals/dke/MalinowskiZ06}
the authors present a classification of different kinds of complex
real-world dimension hierarchies, and they define the MultiDimER
model for the conceptual design of complex MD models based on an
extension of the well-known Entity-Relationships (ER) model. The
idea is that this classification guides developers to properly
capture at a conceptual level the precise semantic of different
kinds of hierarchies without being limited by current data analysis
tools. Modeling of so-called generalized hierarchies is allowed, but
somewhat surprisingly the authors do not consider generalization
constructs (although extended ER models typically include support
for generalization relationships). Furthermore, the authors discuss
how to map these conceptual hierarchies to the relational model
(enabling implementation in commercial tools). However, the mapping
between the conceptual and the logical is described informally. In
addition, the commented mapping is tool-dependent and it may vary
depending on the scenario. Finally, each mapping is defined
independently from the others and the combination of multiple
mappings in a process is not addressed at all. In particular, it
remains open which mapping needs to be applied first if several of
them are applicable (e.g., if a hierarchy is at the same time
roll-up incomplete and non-strict, and both kinds of heterogeneity
need to be resolved). As no ordering for applying mappings for
different hierarchies is specified, applicability problems of the
overall approach arise.

Similarly, in~\cite{DBLP:conf/dawak/MansmannS06,MansmannIJDWDM07}
the authors argue that OLAP tools could fail when dealing with
complex hierarchies for real-world situations, since they only admit
homogeneous dimension hierarchies. Hence, hierarchies need to be
modeled precisely at a conceptual level and then complex hierarchies
should be transformed to make them navigable in a uniform manner. To
this end, the authors present a framework for conceptual modeling of
complex hierarchies and their transformation into a set of
well-behaved sub-hierarchies without summarizability problems. They
present how to deal with generalization hierarchies at a conceptual
model by using informal guidelines, and they use the algorithms from
Pedersen et al.~\cite{DBLP:conf/vldb/PedersenJD99} (slightly
modified) to eliminate roll-up/drill-down incomplete and non-strict
hierarchies at the instance level. They focus on visualization of
data and every of the proposed transformation aims at incorporating
a different kind of hierarchies into a visual OLAP interface to
query complex data properly.

In the same spirit, in~\cite{DBLP:conf/er/AkokaCP01} the authors
argue that modeling hierarchies directly at the logical level (by
using, for example star or snowflake schemas) can be misleading;
hence careful conceptual design is necessary, which then requires a
non-trivial transformation to derive a logical representation. The
authors advocate the use of aggregation and generalization
associations to model hierarchies in UML. Nevertheless, they see
problems with the use of generalization in hierarchies in MD models,
and they aim to preserve the information contained in UML
generalizations by transforming them into aggregations following the
proposal of~\cite{DBLP:conf/dmdw/MoodyK00}. The transformations are
formalized with OCL rules in~\cite{DBLP:journals/dss/PratAC06}.


\subsection{Fact-Dimension Relationships}
Few efforts address the proper design of relationships between facts
and dimensions and their summarizability issues. Surprisingly, every
work is only concerned with the many-to-many relationships between
facts and dimensions, i.e. non-strictness, thus ignoring incomplete
relationships.

The first proposal considers so-called multivalued
dimensions~\cite{book/Kimball/DW}, which permit a star schema to
have non-strict relationships between facts and dimensions by means
of a bridge table. This bridge table captures the non-strict
fact-dimension relationship by using foreign keys that refer to the
tables that represent the dimension and the fact. These foreign keys
also form a compound primary key for the bridge table. Song et
al.~\cite{DBLP:conf/dmdw/SongRME01} focus on defining several
methods to improve the use of a bridge table. They advocate the
representation of many-to-many relationships with correct semantics,
maintaining at the same time the star schema structure by defining
six different approaches. They also give advantages and
disadvantages of each approach and recommendations for their use.
Apart from summarizability issues, the authors consider other
challenges such as storage or performance requirements.
Unfortunately, both
approaches~\cite{book/Kimball/DW,DBLP:conf/dmdw/SongRME01} are
defined at the logical level, which requires a lot of expertise to
model real-world situations in too complex schemas.

Pedersen et
al.~\cite{DBLP:journals/is/PedersenJD01,DBLP:conf/vldb/PedersenJD99}
state that non-strict relationships between facts and dimensions are
necessary in many real-world situations, therefore, these
relationships must be directly captured in a conceptual model.
Nevertheless, in~\cite{DBLP:conf/vldb/PedersenJD99} summarizability
is tackled at the instance level by modifying data instances, which
raises several problems as mentioned in
Sect.\ref{a1:sec:dim-hierarchies}. Another point of view is
described in~\cite{DBLP:journals/is/PedersenJD01}, where a
relational approach is described for representing fact-dimension
non-strict\-ness by means of alternatives that broaden the use of
bridge tables with more expressive solutions.


\section{Discussion}
\label{a1:sec:discussion} Early approaches for MD modeling exhibited
a lack of rich mechanisms to specify different kinds of complex MD
structures. For example, the well-known star
schema~\cite{book/Kimball/DW} does not explicitly define dimension
hierarchies, while other approaches only consider conceptual MD
models with limited expressiveness (such
as~\cite{DBLP:journals/ijcis/GolfarelliMR98,DBLP:conf/dmdw/HusemannLV00}),
where summarizability is guaranteed since problematic complex
structures are ignored. However, this lack of modeling support
increases the modeling efforts necessary to reflect complex
real-world scenarios.

Due to this fact, several
approaches~\cite{DBLP:journals/is/PedersenJD01,DBLP:conf/caise/MalinowskiZ04,DBLP:journals/tods/HurtadoGM05,DBLP:conf/dawak/MansmannS06}
arose to define more expressive MD formalisms for modeling complex
real-world scenarios, whilst ensuring summarizability. The common
foundation of these works is the definition of a classification of
different kinds of complex MD structures in order to ease the task
of designers about identifying different real-world situations. Most
of these approaches present a set of informal guidelines to
transform the defined complex MD structures into MD structures which
enforce summarizability but which require manual decisions and a lot
of expertise when dealing with complex structures, which reduces
their applicability. Furthermore, these approaches do not provide
enough expressivity to specify every complex MD structure, thus only
providing partial solutions. Hence, subsequent research dealt with
the definition of more expressive MD formalisms for defining
real-world
scenarios~\cite{DBLP:journals/dke/Lujan-MoraTS06,DBLP:journals/is/AbelloSS06,DBLP:journals/dss/PratAC06}.
These approaches are based upon an object-oriented approach to allow
designers to model more complex MD structures, but they neither
offer guidelines for using those more expressive features properly
nor formal mechanisms to avoid summarizability problems of complex
MD models. Actually, even full expressiveness of object-orientation
(generalization) is not exploited so far for considering
summarizability on MD modeling.

Therefore, the discussion revolves around two key issues in MD
modeling~\cite{DBLP:conf/dolap/RizziALT06}: defining complex
structures in an explicit way, and giving mechanisms to support
their implementation to avoid the semantic gap regarding
summarizability problems. Moreover, suggestions for improving the
state-of-the-art are also provided.


\subsection{Modeling Complex Multidimensional Structures}
Several properties related to the definition of MD models are
addressed in Table~\ref{a1:tab:related1}. Specifically, this table
focuses on showing (i) the technique or notation used to specify the
MD model (technique column), (ii) the supported kind of hierarchies
according to our unified notation (roll-up incompleteness,
drill-down incompleteness, non-strictness, and generalization
columns), (iii) the different kinds of relationships between a fact
and a dimension (incomplete and non-strict fact-dimension
relationships), and (iv) some kind of guidelines or classification
framework provided in support of modeling complex MD structures. At
first sight, state-of-the-art in MD modeling lacks an overall
approach that defines mechanisms to guide designers to define every
complex MD structure, since current research only offers partial
solutions either for dimension hierarchies or for fact-dimension
relationships.

\begin{table}
    \tiny
    \centering
        \caption{Properties considered for modeling complex MD structures}
    \label{a1:tab:related1}
        \begin{tabular}{|c|c|c|c|c|c|c|c|c|}
        \hline
        & Technique & \multicolumn{4}{|c|}{Hierarchies} & \multicolumn{2}{|c|}{F-D relationships} & Guidelines \\
        \cline{3-8}
        & & Drill-down inc. & Roll-up inc. & Non-strictness & Generalization & Incompleteness & Non-strictness & \\
        \hline
        \hline
        \cite{DBLP:journals/dke/Lujan-MoraTS06} & UML & Yes & Yes & Yes & Yes & Yes & Yes & No \\
        \cite{DBLP:journals/is/AbelloSS06} & UML & No & No & Yes & Yes & Yes& Yes & No \\
        \cite{DBLP:conf/er/AkokaCP01,DBLP:journals/dss/PratAC06} & UML & No & Yes & Yes & Yes & No & No & No \\
        \cite{DBLP:conf/caise/MalinowskiZ04,DBLP:journals/dke/MalinowskiZ06} & ER & Yes & Yes & Yes & Yes & No & No & Yes \\
        \cite{DBLP:conf/vldb/PedersenJD99,DBLP:journals/is/PedersenJD01} & Formalism & Yes & Yes & Yes & No & No & Yes & Yes \\
        \cite{DBLP:conf/dawak/MansmannS06,MansmannIJDWDM07} & ER & Yes & Yes & Yes & Yes & No & Yes & Yes \\
        \cite{DBLP:journals/ijcis/GolfarelliMR98} & DFM & No & No & No & No & No & No & No \\
        \cite{DBLP:conf/dmdw/HusemannLV00,DBLP:journals/is/LechtenborgerV03} & MNFs & No & Yes & No & Yes & No & No & Yes \\
        \cite{book/Kimball/DW} & Relational & No & No & No & No & No & Yes & Yes \\
        \cite{DBLP:journals/tods/HurtadoGM05} & Formalism & Yes & Yes & No & No & No & No & Yes \\
%        \cite{DBLP:conf/ssdbm/LenzS97} & Formalism & Yes & Yes & No & - & Yes & No & Yes --> check again!!!\\
        \cite{DBLP:conf/dmdw/SongRME01} & Relational & No & No & No & No & No & Yes & Yes \\
        \hline
        \end{tabular}
\end{table}

\subsubsection{Techniques} The most popular techniques used to define
MD structures are somehow related to a relational viewpoint. Thus,
some approaches~\cite{book/Kimball/DW,DBLP:conf/dmdw/SongRME01}
directly define MD structures as relational concepts (such as
tables, columns keys, and so on) at the logical level. Other
approaches~\cite{DBLP:journals/ijcis/GolfarelliMR98,DBLP:conf/dmdw/HusemannLV00}
define several MD structures at a conceptual level by using their
own notation, but they are highly influenced by a subsequent
relational implementation, since they use well-known concepts from
relational databases such as functional dependencies. Moreover,
other conceptual
approaches~\cite{DBLP:journals/dke/MalinowskiZ06,MansmannIJDWDM07}
are based on Entity-Relationship modeling, which is suitably
extended by additional notation to accommodate specifics of
multidimensional modeling.
%
% The problem is that models highly related to a relational
% implementation are not as expressive as required to model every
% complex MD structure at the conceptual level. To overcome this
% problem, other
Finally, some approaches use either the
UML~\cite{DBLP:journals/dke/Lujan-MoraTS06,DBLP:journals/is/AbelloSS06,DBLP:journals/dss/PratAC06}
or their own
formalisms~\cite{DBLP:journals/tods/HurtadoGM05,DBLP:journals/is/PedersenJD01}.

In any case, it is apparent that a highly expressive modeling
technique or language must be used to be able to reflect any
real-world situation to model complex MD structures in an easy way
at the conceptual level.

\subsubsection{Hierarchies}
Approaches at the logical level such as~\cite{book/Kimball/DW} fail
in providing mechanisms to model different kinds of hierarchies.
Although, the approach proposed
in~\cite{DBLP:journals/ijcis/GolfarelliMR98} is defined at the
conceptual level, it only considers regular hierarchies. The
conceptual approach defined
in~\cite{DBLP:conf/dmdw/HusemannLV00,DBLP:journals/is/LechtenborgerV03}
increases the level of expressiveness, by allowing to model roll-up
incomplete and generalization hierarchies,
while~\cite{DBLP:journals/tods/HurtadoGM05} only addresses roll-up
and drill-down incompleteness.

Thanks to the expressiveness of UML the approach described
in~\cite{DBLP:journals/dke/Lujan-MoraTS06} covers every possible
kind of hierarchy. Other UML-based approaches lack some features:
\cite{DBLP:journals/is/AbelloSS06} does not deal with roll-up and
drill-down incompleteness, while \cite{DBLP:journals/dss/PratAC06}
does not give support for drill-down incompleteness.

Some non-UML approaches are also very expressive. The proposal
of~\cite{DBLP:journals/is/PedersenJD01} only misses generalization
relationships within a dimension hierarchy, while
in~\cite{DBLP:journals/dke/MalinowskiZ06} a form of generalization
is considered but without using generalization constructs. Finally,
in~\cite{MansmannIJDWDM07}, every kind of dimension hierarchy is
addressed.

Hence, few approaches are able to represent every kind of dimension
hierarchy to model every possible real-world situation, which must
be a desirable property for any MD modeling approach.

\subsubsection{Fact-dimension Relationship} Several approaches point out the necessity for
support of many-to-many relationships between fact and dimensions
(non-strictness), e.g.,~\cite{DBLP:journals/dke/Lujan-MoraTS06}.
However, regarding incompleteness, there are only two works that
allow its
definition~\cite{DBLP:journals/dke/Lujan-MoraTS06,DBLP:journals/is/AbelloSS06}.
Other works are somehow related to this issue, e.g.,
\cite{DBLP:conf/vldb/PedersenJD99}~stresses the necessity for having
facts with at least one dimension value in every dimension in order
to avoid complex and misleading models. Then, incomplete
relationships between facts and dimensions are not allowed, and
certain real-world situations cannot be specified.

\subsubsection{Guidelines} Apart from a specific notation, several approaches provide guidance for defining complex MD
structures. A classification framework of different kinds of
dimension hierarchies is defined
in~\cite{DBLP:journals/dke/MalinowskiZ06,DBLP:journals/is/PedersenJD01,MansmannIJDWDM07}
in order to help designers to discover situations in the real-world
that can be modeled according to a certain dimension hierarchy type.
Other work~\cite{DBLP:conf/dmdw/HusemannLV00} defines a process to
apply multidimensional normal forms in modeling dimension
hierarchies. On the other hand, thanks to the definition of
dimension constraints in~\cite{DBLP:journals/tods/HurtadoGM05},
summarizability can be characterized and checked. Finally, how to
use bridge tables for designing non-strict dimensioning is proposed
by~\cite{book/Kimball/DW}. Furthermore, there are several proposals
that broaden the use of bridge tables with more expressive
solutions~\cite{DBLP:journals/is/PedersenJD01,DBLP:conf/dmdw/SongRME01}.


\subsection{Ensuring Summarizability in Multidimensional Models}
In order to check and enforce summarizability in MD models,
state-of-the-art offers different kinds of approa\-ches.
Table~\ref{a1:tab:related2} focuses on providing the properties
related to the transformation mechanisms between complex MD
structures and their counterparts without summarizability problems.
The following properties have been studied: (i) mechanism used to
check the summarizability conditions (none, guidelines, algorithm,
rules) in the kind column, (ii) level of automation (manual,
semiautomatic, automatic), (iii) level in which the summarizability
is enforced (schema, instance), and (iv) tool support for
summarizability fulfillment.

\begin{table}
    %\tiny
    \centering
       \caption{Properties considered for ensuring summarizability}
    \label{a1:tab:related2}
        \begin{tabular}{|c|c|c|c|c|}
        \hline
        & Kind  & Automation & Level & Tool\\
        \hline
        \hline
        \cite{DBLP:journals/dke/Lujan-MoraTS06} & none & - & - & - \\
        \cite{DBLP:journals/is/AbelloSS06} & none & - & - & - \\
        \cite{DBLP:conf/er/AkokaCP01,DBLP:journals/dss/PratAC06} & Rules & Semiautomatic & Schema/instance & No \\
        \cite{DBLP:conf/caise/MalinowskiZ04,DBLP:journals/dke/MalinowskiZ06} & Guidelines & Manual & Schema/instance & No \\
        \cite{DBLP:conf/vldb/PedersenJD99,DBLP:journals/is/PedersenJD01} & Algorithm & Automatic & Instance & Yes \\
        \cite{DBLP:conf/dawak/MansmannS06,MansmannIJDWDM07} & Algorithm & Semiautomatic & Schema/instance & Yes \\
        \cite{DBLP:journals/ijcis/GolfarelliMR98} & Direct & - & - & - \\
        \cite{DBLP:conf/dmdw/HusemannLV00,DBLP:journals/is/LechtenborgerV03} & Rules & Automatic & Schema/instance & No \\
        \cite{book/Kimball/DW} & Direct & - & - & - \\
        \cite{DBLP:journals/tods/HurtadoGM05} & Algorithm & Automatic & Schema &  Yes \\
%        \cite{DBLP:conf/ssdbm/LenzS97} & Guidelines & Manual & Schema/instance &  No \\
        \cite{DBLP:conf/dmdw/SongRME01} & Direct & - & - & - \\
        \hline
        \end{tabular}
\end{table}

\subsubsection{Kind}
Importantly, at one end of the spectrum some of the studied
approaches lack mechanisms to check summarizability constraints:
those approaches only address how to model complex MD structures
(``none'' in Tab.~\ref{a1:tab:related2}). At the other end, there
are approaches that focus on directly defining a
summarizability-compliant model, so they do not need to check
summarizability conditions (``direct'' in
Tab.~\ref{a1:tab:related2}). In between, we can find approaches that
either use informal guidelines to help designers check
summarizability~\cite{DBLP:journals/dke/MalinowskiZ06} or a set of
rules or algorithms to formalize summarizability
checking~\cite{DBLP:conf/dmdw/HusemannLV00,DBLP:journals/is/PedersenJD01,MansmannIJDWDM07,DBLP:journals/dss/PratAC06,DBLP:journals/tods/HurtadoGM05}.

\subsubsection{Automation}
The kind of used modeling technique somehow determines the level of
automation. Therefore, the proposals that are automatic are based on
algorithms~\cite{DBLP:journals/is/PedersenJD01,DBLP:journals/tods/HurtadoGM05,DBLP:conf/dmdw/HusemannLV00}.
Manual or semiautomatic proposals are based on informal guidelines
which prevent their total
automation~\cite{DBLP:journals/dke/MalinowskiZ06}. Other approaches
are based on rules \cite{DBLP:journals/dss/PratAC06} or algorithms
\cite{MansmannIJDWDM07} but they need human interaction to validate
their application or to provide further information (e.g.
determining the applicable aggregation functions). Moreover, the
level of automation is also influenced by the complexity in applying
the corresponding mechanisms. The proposals described
in~\cite{DBLP:journals/dke/MalinowskiZ06,MansmannIJDWDM07} addresses
summarizability for entire dimension hierarchies; however, there is
 no formal mechanism to check summarizability constraints in the hierarchy as
 a whole, which can include different features at the same time.
Other approaches such
as~\cite{DBLP:journals/is/PedersenJD01,DBLP:journals/tods/HurtadoGM05}
consider relationships between dimension levels to obtain a
summarizable version of the dimension hierarchy. This last solution
also allows the definition of modular and easy-to-apply algorithms
that can be applied automatically.

\subsubsection{Level} Summarizability checking and enforcing can be done
at two different levels. Some approaches directly transform the data
instances to ensure
summarizability~\cite{DBLP:journals/is/PedersenJD01}, e.g., adding
some special values, or requiring information from data
instances~\cite{DBLP:conf/dmdw/HusemannLV00,DBLP:conf/dmdw/SongRME01}.
The advantage of these approaches lies in their algorithmic
mechanisms to check summarizability; however, non-trivial effort is
required to preprocess the huge amounts of data instances before
checking the summarizability conditions. Other approaches combine
the instance level with information extracted from the schema to
decrease the level of required preprocessing and improve
performance~\cite{DBLP:journals/dss/PratAC06,MansmannIJDWDM07,DBLP:journals/dke/MalinowskiZ06}.
Anyway, the most desirable situation is working only with
information from the schema, as stated
by~\cite{DBLP:journals/tods/HurtadoGM05} (their dimension
constraints allow us to test summarizability at the schema level),
which avoids exploring potentially huge data instances.

\subsubsection{Tool Support}
Only three approaches present some kind of implementation that helps
in checking and enforcing the summarizability conditions. The
approach for modeling and transforming complex MD structures
of~\cite{MansmannIJDWDM07} was implemented as part of an OLAP tool
in order to improve the visualization expressiveness. The algorithms
for enforcing summarizability proposed
in~\cite{DBLP:journals/tods/HurtadoGM05,DBLP:journals/is/PedersenJD01},
were implemented and tested. Nevertheless, it is worth noting that,
even though several implementations are described, there is neither
a prototype nor a tool that supports checking and enforcing the
summarizability conditions in a modular and easy-to-use fashion for
designers.

\subsection{Suggestions for Future Work}
There are three fundamental areas that need to be covered: (i)
ensuring summarizability in a comprehensive DW design process, (ii)
addressing every complex MD structure in an integrated way, and
(iii) developing a tool that supports the design of complex MD
structures without summarizability problems. In addition, DWs are
more and more used in other novel areas, such as biological,
multimedia, or spatio-temporal~\cite{DBLP:conf/dolap/RizziALT06},
rather than the classical enterprise domain. Interestingly, this
scenario poses new research challenges for tackling summarizability,
since even if a carefully designed MD model is obtained,
summarizability may not be ensured at all without considering
specific semantics of each domain. For example, geographical DWs
should use spatial semantics to face up with partial containment
dimension hierarchies~\cite{DBLP:journals/vldb/JensenKPT04}, as well
as imprecise and uncertain
data~\cite{DBLP:journals/vldb/BurdickDJRV07}. Therefore, specific
semantics of each domain should be further investigated to help
ensuring summarizability in the next generation of DWs.

%In addition, the next generation of DWs poses new research
%challenges for tackling summarizability, since DWs are more and more
%used away from the classical enterprise domain, in unusual areas,
%such as biological, multimedia, or spatio-temporal
%applications~\cite{DBLP:conf/dolap/RizziALT06}. Even if a carefully
%designed MD model is obtained, summarizability may not be ensured at
%all in these emerging domains without considering their specific
%semantics, e.g., geographical DWs should use spatial semantics to
%face up with partial containment dimension
%hierarchies~\cite{DBLP:journals/vldb/JensenKPT04}, as well as
%imprecise and uncertain
%data~\cite{DBLP:journals/vldb/BurdickDJRV07}. Therefore, specific
%semantics of each domain should be further investigated to help
%ensuring summarizability in the next generation of DWs.

%Some of the new summarizability issues will come from facing up with
%the specific semantics of these emerging domains, since they involve
%more complex data types than traditional and simple alphanumerical
%data types, e.g., specialized spatial data structures used in GIS
%(Geographic Information Systems) or data streams related to Business
%Process Monitoring (BPM).

Summarizability needs to be addressed as integrated aspect of a
comprehensive design process (rather than as problem concerning
isolated MD concepts). A good way to achieve this goal is to follow
the con\-cep\-tual/lo\-gi\-cal/phy\-si\-cal design phases, which
allows the designer to take advantage of complex MD structures at a
conceptual level without taking summarizability into account for the
initial design.  Afterwards, this rich conceptual model should be
transformed (semi-) automatically into another model based on
summarizability considerations (e.g., non-strict dimension-fact
relationships may be replaced with more complex schema structures
that explicitly represent different granularities). Whether this
transformation should work at the conceptual level or should lead to
another model at the logical level is subject to future work. In any
case, the transformed model then serves as basis for an
implementation of the MD model without summarizability problems,
thus bridging the inherent semantic gap.

In addition, while current work gives more importance to dimension
hierarchies, we are not aware of research that (a) integrates the
definition of all types of complex MD structures presented in this
survey and (b) at the same time defines guidelines to ease the task
of designer.  For example, non-strict relations between facts and
dimensions do occur in real-world scenarios, hence need to be
modeled conceptually and then transformed correctly into their
corresponding implementation, addressing summarizability
problems~\cite{DBLP:conf/dmdw/SongRME01,DBLP:journals/is/PedersenJD01}.
Furthermore, we argue that object-based approaches are a good choice
to proceed: Importantly, approaches that use UML
constructs~\cite{DBLP:journals/is/AbelloSS06,DBLP:journals/dke/Lujan-MoraTS06,DBLP:journals/dss/PratAC06}
can represent generalization relationships explicitly to ensure
context-sensitive summarizability.  Thus, there is no need for
designers to explicitly deal with dimension
constraints~\cite{DBLP:journals/tods/HurtadoGM05} or context
dependencies~\cite{DBLP:conf/dmdw/HusemannLV00,DBLP:journals/is/LechtenborgerV03},
while retaining their advantages.

Finally, advanced modeling and transformation approaches need to be
supported via tools that help designers in checking and enforcing
summarizability.  Such support is even more pressing when dealing
with complex MD models.

\section{Conclusions}
\label{a1:sec:conclusions} MD modeling stresses the definition of
complex MD structures, allowing designers to deal with real-world
situations, such as roll-up incomplete or non-strict hierarchies.
Specifically, powerful modeling constructs must be used to define
rich conceptual MD structures, such as (i) relationships between
levels within a dimension hierarchy, and (ii) relationships between
facts and dimensions.

One key issue when a MD model is being defined is dealing with
summarizability. Summarizability guaranties correct aggregation of
data. However in real-world scenarios summarizability does not arise
in a natural way, and frequently MD design starts with
non-summarizable but easily specified and understood conceptual
models. Then, an equivalent summarizable model must be obtained
before the implementation.

This survey provides researchers with an overall understanding about
current approaches for modeling complex MD structures that reflect
real-world situations and the mechanisms for enforcing
summarizability.  We conclude that further research is needed to
tackle summarizability issues in MD modeling in a comprehensive way
which remains as an open research problem.

%\bibliographystyle{spbasic}      % basic style, author-year citations
%\bibliographystyle{spmpsci}      % mathematics and physical sciences
%\bibliographystyle{spphys}       % APS-like style for physics

%\bibliographystyle{abbrv}

\bibliographystyle{abbrv}
\bibliography{tesis}


%
