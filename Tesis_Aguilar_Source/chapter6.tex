%%%%%%%%%%%%%%%%%%%%% chapter.tex %%%%%%%%%%%%%%%%%%%%%%%%%%%%%%%%%
%
% sample chapter
%
% Use this file as a template for your own input.
%
%%%%%%%%%%%%%%%%%%%%%%%% Springer-Verlag %%%%%%%%%%%%%%%%%%%%%%%%%%

\chapstarthook{The content of this chapter corresponds with the
following paper: \textbf{J.-N. Maz{\'o}n and J. Trujillo. A Model
Driven Modernization Approach for Automatically Deriving
Multidimensional Models in Data Warehouses. 26th International
Conference on Conceptual Modeling (ER 2007), Auckland (New Zealand),
November 5-9, 2007. Lecture Notes in Computer Science Vol. 4801,
pp.56--71. [Acceptance rate: 22.2\%].}}


\chapter{A Model Driven Modernization Approach for Automatically Deriving Multidimensional Models in Data Warehouses}
\label{c6} % Always give a unique label
% use \chaptermark{}
% to alter or adjust the chapter heading in the running head

The development of a data warehouse requires an in-depth analysis of
data sources. In the previous chapter, it is assumed that
documentation of the data sources is available. However, this is not
always true, since in real scenarios data sources are, in reality,
legacy systems and their manual analysis may be extremely difficult.
In order to overcome these problems, this chapter considers the
development of a data warehouse as a modernization scenario which
addresses the analysis of the available data sources, thus
discovering multidimensional structures with which to either derive
a data-driven conceptual multidimensional model or to reconcile a
requirement-driven conceptual multidimensional model with data
sources. The content of this chapter corresponds with the part of
the approach shaded in the figure below.

\begin{figure}[h!]
  \begin{center}
    \includegraphics[width=0.7\textwidth]{img/chapters/chapter6}
  \end{center}
  %\caption{} \label{}
\end{figure}


The content of this chapter was published in the \emph{International
Conference on Conceptual Modeling (ER)}. This is one of the  most
important conferences in data and process modeling, database
technology, and database applications. This conference is a wide
forum for researchers and industrial experts interested in all
aspects of database and information systems design and usage. Topics
of interest include data warehousing and business intelligence.
\emph{ER} is a top-ranking conference, since it has an
\emph{Estimated Impact of Conference (EIC)} value of \emph{0.91}
according to \emph{The Computer Science Conference Ranking Website}
(\url{http://www.cs-conference-ranking.org/home.html}). The
\emph{acceptance rate} of this conference is usually around
\emph{20\%}.


\includepdf[openright=true,pages={1-16}]{mazon_er07.pdf}
%
