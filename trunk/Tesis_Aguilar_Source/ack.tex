
\preface[Agradecimientos]

Un trabajo de investigaci�n nunca es un esfuerzo individual, sino
que se realiza siempre con el apoyo y la ayuda de muchas personas,
tanto en lo profesional como en lo personal. Imposible enumerarlas a
todas, as� que a todas ellas: gracias. Por el apoyo en el trabajo,
por esa charla delante de un caf�, por esa ayuda con el ingl�s, por
esa palmada en la espalda en los d�as malos, por esos desayunos
temprano por la ma�ana, por esos momentos de desconexi�n haciendo
deporte, por esas comidas diarias con \emph{tupper} incluido, por
ser el faro en la noche\dots Por todo.

A Inma, por convertir los malos momentos en una sonrisa. Este
trabajo es tanto tuyo como m�o.

A toda mi familia por hacerme sentir especial, en particular a mis
padres porque todo lo que soy es gracias a ellos.

A Juan Carlos Trujillo, por todo lo compartido en este tiempo y por
todo el apoyo, decisivo y fundamental, para seguir adelante.

A los que me ``sufrieron'' m�s directamente en el trabajo diario:
Jes{\'u}s Pardillo y Octavio Glorio, con los que he colaborado
estrechamente en la realizaci�n de varios trabajos de investigaci�n,
as� como en la implementaci�n de la herramienta de modelado que se
muestra en esta tesis. Por los momentos compartidos en estos a�os y
por los momentos que quedan por venir\dots

A Emilio Soler, por todos estos a�os de ``conexi�n cubana'' y por
esta recta final de trabajo intenso, codo con codo.

A Luisa Mic� y Jose Manuel I\~{n}esta porque con ellos empez� mi
periodo de becario en el Departamento de Lenguajes y Sistemas
Inform�ticos de la Universidad de Alicante.

A mis compa�eros del grupo de investigaci�n Lucentia del
Departamento de Lenguajes y Sistemas Inform�ticos de la Universidad
de Alicante, Cristina Cachero, Jose Jacobo Zubcoff, Sergio Luj�n,
Rafael Romero y Lily Mu�oz por las discusiones de trabajo y toda la
colaboraci�n que me brindan.

A los miembros del grupo de investigaci�n IWAD del Departamento de
Lenguajes y Sistemas Inform�ticos de la Universidad de Alicante, en
especial a Jaime G�mez, Irene Garrig�s y Santiago Meli� por el apoyo
recibido durante este per�odo de investigaci�n.

A todos los miembros del grupo de investigaci�n GPLSI del
Departamento de Lenguajes y Sistemas Inform�ticos de la Universidad
de Alicante por todos los momentos compartidos.

Tambi�n quisiera expresar mi m�s profundo agradecimiento al Prof.
Dr. Gottfried Vossen y al Dr. Jens Lechtenb�rger por facilitarme la
realizaci�n de una estancia de investigaci�n en la Universidad de
M�nster durante el verano de 2007. Su inter�s en mi trabajo, sus
valiosas discusiones y su apoyo han tenido gran influencia en mi
investigaci�n.

Finalmente, tambi�n quisiera agradecer al Prof. Dr. Eric Yu y al Dr.
Jordi Cabot su apoyo durante mi estancia en la Universidad de
Toronto durante el verano de 2008.
