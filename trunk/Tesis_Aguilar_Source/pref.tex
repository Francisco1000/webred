%%%%%%%%%%%%%%%%%%%%%% pref.tex %%%%%%%%%%%%%%%%%%%%%%%%%%%%%%%%%%%%%
%
% sample preface
%
% Use this file as a template for your own input.
%
%%%%%%%%%%%%%%%%%%%%%%%% Springer-Verlag %%%%%%%%%%%%%%%%%%%%%%%%%%

\preface

Designing a multidimensional model of a data warehouse is a highly
complex, prone to fail, and time consuming task, due to the fact
that (i) the information needs of decision makers and the available
operational data sources that will populate the data warehouse must
both be considered together in a conceptual multidimensional model,
and (ii) summarizability-aware non-trivial mappings must be
performed to obtain the final implementation of this conceptual
multidimensional model. However, no significant effort has been made
to take these issues into account in a systematic, well structured
and comprehensive development process. To overcome the lack of such
a process, this PhD Thesis proposes a model-driven approach for the
development of a hybrid multidimensional model at the conceptual
level and for the automatic derivation of its logical representation
as a basis of implementation. A normalization process is also
proposed to avoid summarizability problems at the conceptual level
and in further implementation stages. Finally, an
\emph{Eclipse}-based tool has been implemented as a proof of concept
of this research. This tool has been used in a case study, which
shows each step of the presented approach.

This PhD Thesis is composed of a set of published and submitted
papers. In order to write this PhD Thesis as a collection of papers,
several requirements must be taken into account as stated by the
University of Alicante. With regard to the content of the PhD
Thesis, it must specifically include a summary which is devoted to
the description of initial hypotheses, research objectives, and the
collection of publications itself, thus justifying its coherence. It
should be underlined that this summary of the PhD Thesis must also
include research results and final conclusions. This summary
corresponds to part~\ref{p1} of this PhD Thesis (chapter~\ref{c1}
has been written in Spanish while chapter~\ref{c2} is in English).

It should be mentioned that this PhD Thesis has been developed
within the PhD program \emph{``Aplicaciones de la Inform�tica''} of
the Department of Software and Computing Systems (\emph{Departamento
de Lenguajes y Sistemas Inform�ticos}, DLSI) of the University of
Alicante. This PhD work was funded by the Spanish Ministry of
Education and Science under the FPU grant AP2005-1360.

Finally, this research was developed under the following projects:
METASIGN (TIN2004-00779) and ESPIA (TIN2007-67078) projects from the
Spanish Ministry of Education and Science; DADASMECA (GV05/220) and
DEMETER (GVPRE/2008/063) projects from the Valencia Ministry of
Enterprise, University and Science (Spain); and MESSENGER
(PCC-03-003-1), DADS (PBC-05-012-2) and QUASIMODO (PAC08-0157-0668)
projects from the Castilla-La Mancha Ministry of Education and
Science (Spain).



%% Please "sign" your preface
\vspace{1cm}
\begin{flushright}\noindent
Alicante, October 2008\hfill {\it Jose Norberto Maz{\'o}n L{\'o}pez}\\
\end{flushright}
