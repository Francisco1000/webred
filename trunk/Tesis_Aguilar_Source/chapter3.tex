%%%%%%%%%%%%%%%%%%%%% chapter.tex %%%%%%%%%%%%%%%%%%%%%%%%%%%%%%%%%
%
% sample chapter
%
% Use this file as a template for your own input.
%
%%%%%%%%%%%%%%%%%%%%%%%% Springer-Verlag %%%%%%%%%%%%%%%%%%%%%%%%%%


\chapstarthook{The content of this chapter corresponds with the
following paper: \textbf{J.-N. Maz{\'o}n and J. Trujillo. An MDA
approach for the development of data warehouses. Decision Support
Systems, 45(1):41--58, 2008. [IF2007: 1.119].}}

\chapter{An MDA Approach for the Development of Data Warehouses}
\label{c3} % Always give a unique label
% use \chaptermark{}
% to alter or adjust the chapter heading in the running head


This chapter describes a standard and integrated model-driven
framework with which to design each component of a data warehouse.
Once this framework is defined, the focus is on the multidimensional
modeling of the data warehouse repository. This chapter specifically
defines a conceptual multidimensional model and how it is translated
to a relational-based logical representation by using a model-driven
approach. The advantages of using the model-driven development to
design data warehouses are also enumerated. The content of this
chapter corresponds with the part of the approach shaded in the
figure below.

\begin{figure}[h!]
  \begin{center}
    \includegraphics[width=0.7\textwidth]{img/chapters/chapter3}
  \end{center}
  %\caption{} \label{}
\end{figure}

The content of this chapter is a paper published in \emph{Decision
Support Systems}. This journal focuses on contributions to the
concepts and operational basis for decision support systems and
techniques for implementing and evaluating decision support systems.
This journal specifically encourages the following topics:
artificial intelligence, data base management, decision theory,
economics, linguistics, management science, mathematical modeling,
amongst others. The common thread of articles published in the
journal is their relevance to theoretical and technical issues for
decision support systems. As data warehousing has been widely
accepted as a key technology through which organizations can improve
their abilities in data analysis and decision support, this journal
is an important forum of publication for research in the data
warehouse domain. Finally, it should be mentioned that this journal
had an \emph{impact factor} of \emph{1.119} in 2007, according to
the \emph{Thomson's Science Citation Index (SCI)}
(\url{http://www.isiwebofknowledge.com/}).


\includepdf[openright=true,pages={1-18}]{mazon_dss08.pdf}



%
