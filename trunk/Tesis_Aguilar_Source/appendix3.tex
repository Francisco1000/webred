%%%%%%%%%%%%%%%%%%%%% chapter.tex %%%%%%%%%%%%%%%%%%%%%%%%%%%%%%%%%
%
% sample chapter
%
% Use this file as a template for your own input.
%
%%%%%%%%%%%%%%%%%%%%%%%% Springer-Verlag %%%%%%%%%%%%%%%%%%%%%%%%%%


\chapstarthook{The content of this chapter has been submitted to the
\emph{ACM Eleventh International Workshop on Data Warehousing and
OLAP, DOLAP 2008}.}


\chapter{Solving Summarizability Problems in Fact-Dimension Relationships for Multidimensional Models}
\label{a3} % Always give a unique label
% use \chaptermark{}
% to alter or adjust the chapter heading in the running head

\chaptermark{Solving summarizability problems in fact-dimension relationships}


Multidimensional analysis allows decision makers to
effi\-ci\-ent\-ly and effectively use data analysis tools, which
mainly depend on multidimensional (MD) structures of a data
warehouse such as facts and dimension hierarchies to explore the
information and aggregate it at different levels of detail in an
accurate way. A conceptual model of such MD structures serves as
abstract basis of the subsequent implementation according to one
specific technology. However, there is a semantic gap between a
conceptual model and its implementation which complicates an
adequate treatment of summarizability issues, which in turn may lead
to erroneous results of data analysis tools and cause the failure of
the whole data warehouse project.  To bridge this gap for
relationships between facts and dimension, we present an approach at
the conceptual level for (i) identifying problematic situations in
fact-dimension relationships, (ii) defining these relationships in a
conceptual MD model, and (iii) applying a normalization process to
transform this conceptual MD model into a summarizability-compliant
model that avoids erroneous analysis of data. Furthermore, we also
describe our \textsc{Eclipse}-based implementation of this
normalization process.


\section{Introduction}
\label{a3:sec:intro} Data analysis tools, such as OLAP (On-Line
Analytical Processing) tools depend on the multidimensional (MD)
structures of a data warehouse that allow analysts to explore,
navigate, and aggregate information at different levels of detail to
support the decision making process. Current approaches for data
warehouse design advocate to start the development by defining a
conceptual model in order to describe real-world situations by using
MD structures~\cite{DBLP:conf/dolap/RizziALT06}. These structures
contain two main elements: On one hand, dimensions which specify
different ways the data can be viewed, aggregated, and sorted (e.g.,
according to time, store, customer, product, etc.). On the other
hand, events of interest for an analyst (e.g., sales of products,
treatments of patients, duration of processes, etc.) are represented
as facts which are described in terms of a set of measures. Every
fact is based on a set of dimensions that determine the granularity
adopted for representing the fact's measures. Dimensions, in turn,
are organized as hierarchies of levels that allow analysts to
aggregate data at different levels of detail. Hence, MD conceptual
modeling must provide mechanisms for defining relationships (i)
between dimensions and facts, and (ii) between levels of aggregation
within a dimension hierarchy. These relationships can be modeled in
a variety of ways in order to reflect real-world situations, and
their accurate yet understandable design is a cornerstone to enable
users to analyze large amounts of data stored in data warehouses to
effectively and efficiently support decision making.

Importantly, a MD model must ensure summarizability, which refers to
the possibility of accurately computing aggregate values with a
coarser level of detail from values with a finer level of detail. If
summarizability is violated, then incorrect results can be derived
in data analysis tools, and therefore erroneous analysis
decisions~\cite{DBLP:conf/ssdbm/LehnerAW98,DBLP:conf/ssdbm/LenzS97}.
Besides, summarizability is a necessary precondition for performance
optimizations based on
pre-aggregation~\cite{DBLP:conf/vldb/PedersenJD99}.

Traditionally, the focus for ensuring summarizability has been on
dimension hierarchies due to the influence of statistical databases
research~\cite{DBLP:conf/ssdbm/LenzS97}. However, within the full
scope of MD modeling in a data warehouse system, summarizability
must be also ensured for fact-dimension relationships, which
surprisingly has been widely ignored so far.

Furthermore, summarizability is usually not addressed at the
conceptual level, but at later stages of the development, e.g., by
using instance-specific transformations of data contained in the
implemented data warehouse~\cite{DBLP:journals/is/PedersenJD01}. We
argue that such data-oriented approaches towards summarizability are
problematic for data warehouse designers and end users as huge
amounts of data need to be transformed and the transformed data
entries need to be interpreted correctly.

Nevertheless, data transformations appear to be attractive at first
sight since summarizability-compliant conceptual models tend to be
more complex and to contain more details than models designed
without taking summarizability into account (the examples presented
throughout this paper will illustrate such details). Moreover, in
\emph{initial} design steps the additional detail provided by
summari\-zability-compliant models may not be necessary at all; in
fact, they may even hinder understandability and communication.

As understandability is among the most important properties of
conceptual models, we argue that conceptual design of MD scenarios
should allow for an initial stage of modeling that ignores
summarizability problems and derives a simplified MD model first.
Then, a normalization process should be applied to transform the
designed MD model into a constrained conceptual model, which is
restricted to those MD structures that do not violate
summarizability. This normalized model contains additional details
and provides a high level of expressiveness in describing real-world
situations.

Bearing these considerations in mind, in this paper, we present an
approach (see Fig.~\ref{a3:fig:approach}) for (i) designing the
different kinds of fact-dimension relationships in a conceptual
model in order to easily and understandably represent real-world
situations regardless of summarizability problems, and (ii) deriving
normalized conceptual models, which are constrained to those
fact-dimension relationships that do not violate summarizability and
which thus serve as basis for the subsequent implementation.

The most important benefit of our approach is that the semantic gap
between conceptual MD models and their implementation in a database
platform is bridged, since an intermediate normalized model is used
to provide a high level of expressiveness in describing MD
structures for real-world situations, while summarizability
conditions are ensured. We point out that in this way, we tackle
summarizability in a platform-independent manner as the normalized
MD model can be easily deployed in any database platform.

\begin{figure}
\begin{center}
\includegraphics[width=0.8\textwidth]{img/dolap08/approach2.png}
\end{center}
\caption{Normalization process to avoid summarizability problems in
MD models} \label{a3:fig:approach}
\end{figure}

The remainder of this paper is structured as follows.  In
Section~\ref{a3:sec:fact_dim1}, we give an overview of
summarizability in MD modeling and present how to model different
kinds of fact-dimension relationships and their summarizability
problems. In Section~\ref{a3:sec:fact_dim2}, we define a
normalization process to ensure summarizability in fact-dimension
relationships, and we describe its implementation in
Section~\ref{a3:sec:implementation}.  We address related work in
Section~\ref{a3:sec:related}, and we provide our conclusions and
sketch future work in Section~\ref{a3:sect:conclusions}.

\section{Fact-Dimension Relationships}\label{a3:sec:fact_dim1}
A crucial decision for designing MD models concerns the grain of the
fact~\cite{book/Kimball/DW}, i.e., the list of dimensions which
defines the scope of the measures in the fact. Therefore, the grain
of the fact is determined by fact-dimension relationships. In this
section, we stress the importance of accurately modeling
fact-dimension relationships, and we describe the different
situations in which summarizability could be violated.

\subsection{Summarizability and MD Modeling}
The notion of \emph{summarizability} was introduced by Rafanelli and
Shoshani~\cite{DBLP:conf/ssdbm/RafanelliS90} in the context of
statistical databases, where it refers to the correct computation of
aggregate values with a coarser level of detail from aggregate
values with a finer level of detail. Lenz and
Shoshani~\cite{DBLP:conf/ssdbm/LenzS97} argue that summarizability
is of most importance for queries concerning MD data, since
violations of this property lead to incorrect aggregation results,
which in turn may lead to erroneous conclusions and decisions.
Although this early work on summarizability is focused on
statistical databases, we consider it as cornerstone in MD modeling,
because the authors lay the foundations for detecting and avoiding
summarizability problems in a MD space.  Specifically, the authors
propose three necessary conditions for summarizability that every
dimension hierarchy must fulfill:(1) \emph{disjointness} and (2)
\emph{completeness} of associations between pairs of dimension
levels imply that every element of the finer dimension level must be
associated with exactly one element of the coarser dimension level,
and (3) \emph{type
  compatibility} is given if a particular aggregate function is applicable to
a given measure for a given set of dimensions.

Although Lenz and Shoshani~\cite{DBLP:conf/ssdbm/LenzS97} only focus
on the relationships between two levels of a dimension hierarchy,
the relationships between facts and dimensions can also cause
summarizability problems in MD modeling of data warehouses. As the
previously-mentioned three conditions are concerned with the proper
definition of dimensions and their hierarchies, they are called
\emph{intradimensional} constraints
by~\cite{DBLP:conf/ssdbm/LehnerAW98}. However, within MD modeling,
\emph{interdimensional} constraints are also required in order to
ensure summarizability~\cite{DBLP:conf/ssdbm/LehnerAW98}. These
interdimensional constraints are related to the grain of the fact in
such a way that, to avoid erroneous results when a MD model is
queried, every measure in the fact must be determined by all
dimensions, which is made formally precise by the first MD normal
form proposed in~\cite{DBLP:journals/is/LechtenborgerV03}.

Intuitively, the relationship between a fact and a dimension must be
many-to-one to avoid summarizability problems, which can be
reflected in the common relational implementation of a star schema,
where the primary key of the fact table is composed of foreign keys
of the dimension tables~\cite{book/Kimball/DW}. Therefore, MD models
are usually defined according to this multiplicity constraint in
order to enforce summarizability in fact-dimension relationships.
However, many-to-one associations between the fact and every
dimension are too strict for certain real-world situations. Indeed,
designers must also deal with scenarios in which different
granularities are necessary and where relationships between a fact
and a dimension can have different multiplicities. For example,
in~\cite{DBLP:conf/dmdw/SongRME01} the authors state that the
relationship between the diagnosis dimension and the billable
patient encounter fact is normally many-to-many, as a patient could
have more than one diagnosis for each billable encounter. However,
incorrect results can be obtained when measures are queried through
such a many-to-many relationship, which indicates summarizability
problems.


\subsection{Classifying Fact-Dimension Relationships}
Modeling different kinds of fact-dimension relationships requires a
highly expressive language. In this paper, we propose to use our UML
(Unified Modeling Language)~\cite{OMG/UML} profile for MD
modeling~\cite{DBLP:journals/dke/Lujan-MoraTS06}. This profile
contains the necessary stereotypes in order to elegantly represent
main MD properties at the conceptual level, thus providing a set of
constructs for modeling real-world MD scenarios\footnote{In this
paper, we focus on an excerpt of this UML profile, and we refer the
reader to~\cite{DBLP:journals/dke/Lujan-MoraTS06} for further
explanations.}.

Specifically, by using our UML profile, the structural properties of
MD modeling are represented by means of a UML class diagram in which
the information is clearly organized into facts and dimensions.
These facts and dimensions are represented by classes stereotyped as
\emph{Fact} (\includegraphics[height=3mm]{img/icons/fact.png}) and
\emph{Dimension}
(\includegraphics[height=3mm]{img/icons/dimension.png})
respectively. \emph{Fact} classes are defined as composite classes
in shared aggregation relationships of many \emph{Dimension}
classes. A fact is composed of measures or fact attributes. These
are represented as attributes with the \emph{FactAttribute}
stereotype (\includegraphics[height=2mm]{img/icons/fa.png}). Our
approach also allows the definition of degenerate dimensions,
thereby representing other fact features in addition to the measures
for analysis. These degenerate dimensions are represented as
stereotyped attributes of the \emph{Fact} class
(\emph{DegenerateDimension} stereotype,
\includegraphics[height=2mm]{img/icons/dd.png}).
Other MD structures that can be defined by using our UML profile are
dimension hierarchies. Each level of a dimension hierarchy is
specified by a \textit{Base} class
(\includegraphics[height=3mm]{img/icons/base.png}) which can contain
dimension attributes. Associations (represented by the stereotype
\textit{Rolls-UpTo},
\includegraphics[height=3mm]{img/icons/roll.png}) between pairs of
\textit{Base} classes form a dimension hierarchy.

In order to cover different situations when the associations between
\emph{Fact} and \emph{Dimension} classes are defined, we take
advantage of the multiplicities in the roles of the \emph{Dimension}
and \emph{Fact} classes (see
Table~\ref{a3:tab:FD-multiplicity-classification}). In practice, to
avoid summarizability problems, there must be a minimum and maximum
multiplicity of $1$ in the end of a \emph{Dimension} which is
related to a \emph{Fact}. In
Table~\ref{a3:tab:FD-multiplicity-classification}, ``regular''
denotes association types without summarizability problems, whereas
the remaining entries indicate some irregularity.

\begin{table}
  \centering
  \caption{Classification of fact-dimension associations}
  \label{a3:tab:FD-multiplicity-classification}
    \begin{tabular}{|c||c|c|c|c|}
         \hline
       & \multicolumn{2}{|c|}{Minimum Multiplicity} & \multicolumn{2}{|c|}{Maximum Multiplicity} \\\hline
       & 0 & 1 & 1 & * \\\hline
      Fact & regular & regular & regular & regular \\
      Dimension & incomplete & regular & regular & non-strict \\\hline
    \end{tabular}
\end{table}

In the following, every possible kind of relationship between facts
and dimensions is described. Several examples are provided by using
our UML profile for MD modeling.

\subsubsection{Regular Fact-Dimension Relationships}
Concerning the regular entries in
Table~\ref{a3:tab:FD-multiplicity-classification}, we first note
that the multiplicities at the fact's end of an association are not
essential when discussing summarizability.  Indeed, dimension
instances may typically occur in zero or more fact instances but
there is no problem if designers know (and model) that dimension
instances occur in at least or at most one fact instance.  In
addition, if the multiplicities at the dimension's end of the
association specify a minimum and maximum multiplicity of $1$ then
the fact is associated with exactly one instance of that dimension.
In this case, the fact's measures are assigned to a uniquely
identified combination of dimension instances, which allows to
change the level of detail within dimensions without summarizability
problems.

\paragraph{Minimum multiplicity of $0$ at the end of the fact} This multiplicity
allows the existence of dimension instances that are not related
with any fact instance. This is the most common option for MD
modeling, e.g., consider a \emph{Product} dimension where some
products have not been sold so far.

\paragraph{Minimum multiplicity of $1$ at the end of the fact} This multiplicity
requires that every dimension instance is related with at least one
fact instance. For practical purposes this multiplicity is usually
ignored, since it introduces additional restrictions in the MD model
that make ETL (Extraction-Transformation-Load) processes more
complex and prone to fail.

\paragraph{Maximum multiplicity of $1$ at the end of the fact} This multiplicity
requires that every dimension instance is related with at most one
fact instance. For practical purposes this multiplicity is usually
ignored as well, since it prevents the orthogonal use of dimensions.
E.g., consider a \emph{Time} dimension where every date can only be
used once.

\paragraph{Maximum multiplicity of $*$ at the end of the fact} This multiplicity
allows dimension instances to be related with many fact instances.
It is the most desirable option within the regular situations.

\subsubsection{Incomplete Fact-Dimension Relationships} An
association between a \emph{Fact} class $f$ and a \emph{Dimension}
class $d$ is \emph{complete} if for every fact instance of $f$,
there exists a dimension instance of $d$ which is related to that
fact instance; otherwise, the association is \emph{incomplete}. This
situation represents a summarizability violation since there is a
granularity mismatch in the instances of the fact. Following our UML
profile, a fact-dimension association is incomplete if the minimum
cardinality at the end of the \emph{Dimension} class is $0$. For
example, the association between the \emph{Customer} dimension and
the \emph{Sales} fact in Fig.~\ref{a3:sample_inc1} exhibits an
incomplete relationship. This example faces the problem of
inconsistent totals, as shown in
Tab.~\ref{a3:tab:DimensioningIncompleteness}, where we assume that
John and Anna buy some products in January, and George goes shopping
in April. The totals arise in a supermarket where some customers
have loyalty cards to get discounts. For those customers, sales are
recorded directly together with their personal information (e.g.,
city of residence). In contrast, sales of (anonymous) customers
without cards are recorded without considering any personal
information. Consequently, when the sales are analyzed by customer
and date some sales are missing (those from anonymous customers).
Only when the customer is not taken into account, the total sales
are correct. The problem of inconsistency is shown in
Tab.~\ref{a3:tab:DimensioningIncompleteness} where some anonymous
sales made in January are not shown when the analysis is performed
along the customer dimension.

\begin{figure}
\begin{center}
\includegraphics[width=0.9\textwidth]{img/dolap08/sample_inc1.png}
\end{center}
\caption{Incomplete relationship between \emph{Sales} fact and
\emph{Customer} dimension} \label{a3:sample_inc1}
\end{figure}


\begin{table}
\centering \caption{Inconsistent totals for sales}
     \label{a3:tab:DimensioningIncompleteness}
\subtable[By customer and time] {
        \label{a3:tab:DimensioningIncompletenessA}
        \begin{tabular}{|c|c|}
        \hline
        Date & Sales \\
        \hline
        \hline
        January-2001 & 25 \\
        April-2001 & 15 \\
        \hline
        Total & 40 \\
        \hline
        \end{tabular}
} \qquad \qquad \subtable[By time] {
     \label{a3:tab:DimensioningIncompletenessB}
        \begin{tabular}{|c|c|c|}
        \hline
        Customer & Date & Sales \\
        \hline
        \hline
        John & January-2001 & 10 \\
        Anna & January-2001 & 5 \\
        George & April-2001 & 15 \\
        \hline
       \multicolumn{2}{|c|}{Total} & 30 \\
        \hline
        \end{tabular}
        }
\end{table}




Designers must be aware of these incomplete fact-dimen\-si\-on
relationships, because they appear in several real-world situations,
e.g., the inherent uncertainty about the function of some genes in
MD models for the biological
domain~\cite{DBLP:journals/ijbra/WangZR05} or the heterogeneous
facts related to surgical processes that can be found in biomedical
data warehouses~\cite{DBLP:conf/er/MansmannNS07}. Otherwise, data
analysis tools will present incorrect results.

\subsubsection{Non-strict Fact-Dimension Relationships}
An association between a \emph{Fact} class $f$ and a
\emph{Dimension} class $d$ is strict if for every instance of the
fact $f$ there exists at most one instance of the dimension $d$
which is related to that fact instance; otherwise, it is called
\emph{non-strict}. Non-strict fact-dimension relationships imply
summarizability problems because for each measure in a fact instance
there could be several instances of the same dimension that are
associated to that measure, thus causing a granularity mismatch. By
using our UML profile, non-strict associations between a \emph{Fact}
class and a specific \emph{Dimension} class are specified by means
of the maximum multiplicity $*$ in the role of the corresponding
\emph{Dimension} class. For example, in
Fig.~\ref{a3:sample_nonstrict1}, the association between the
\emph{Sales} fact and the \emph{Salesperson} (SP) is non-strict,
which means that more than one salesperson may be involved in the
same sale. This situation requires special care to avoid the double
counting problem, i.e., a measure in the fact is considered more
than once when the data is analyzed, thus producing erroneous
results. This problem is illustrated in
Tab.~\ref{a3:tab:DimensioningNonStrictness}: As the sale made on
January 17 is shared by Bill and Peter, it should be counted once
but it will be counted twice in naive implementations.

\begin{figure}
\begin{center}
\includegraphics[width=0.8\textwidth]{img/dolap08/sample_nonstr1_date.png}
\end{center}
\caption{Non-strict relationship between \emph{Sales} fact and
\emph{Salesperson} dimension} \label{a3:sample_nonstrict1}
\end{figure}


\begin{table}
\centering \caption{Double counting problem for sales aggregated by
salesperson}
     \label{a3:tab:DimensioningNonStrictness}
\subtable[As single sales] {
        \label{a3:tab:DimensioningNonStrictnessA}
        \begin{tabular}[t]{|c|c|c|}
        \hline
        Date & SP & Sales \\
        \hline
        \hline
        17/01/07 & Bill & 10 \\
        17/01/07 & Peter & 10 \\
        18/01/07 & Bill & 5 \\
        18/01/07 & Peter & 5 \\
        \hline
        \multicolumn{2}{|c|}{Total} & 30 \\
        \hline
        \end{tabular}
} \qquad \qquad \subtable[As joint sales] {
     \label{a3:tab:DimensioningIncompletenessB}
        \begin{tabular}[t]{|c|c|c|}
        \hline
        Date & SP & Sales \\
        \hline
        \hline
        17/01/07 & Bill, Peter & 10 \\
        18/01/07 & Bill & 5 \\
        18/01/07 &  Peter & 5 \\
        \hline
        \multicolumn{2}{|c|}{Total} & 20 \\
        \hline
        \end{tabular}
        }
\end{table}



Non-strict fact-dimension relationships appear in a pletho\-ra of
real-world situations, such as the relationships between bank
customers and accounts~\cite{book/Kimball/DW}, between insured
drivers and policyholders~\cite{book/Kimball/DW}, or between
patients and
diagnoses~\cite{book/Kimball/DW,DBLP:journals/is/PedersenJD01,DBLP:conf/dmdw/SongRME01}.


\section{Normalization}\label{a3:sec:fact_dim2}
A normalization process is carried out in order to obtain a MD model
that ensures summarizability while accurately capturing the
expressiveness of the demanded real-world situation. The output of
such a process is a conceptual model constrained to those elements
and relationships that do not violate summarizability. From this
normalized MD model, an implementation which ensures consistent
results in data analysis tools can be obtained.

Our normalization process is performed at the conceptual level
(recall Fig.~\ref{a3:fig:approach}) by using schema information to
ensure summarizability. To this end, a normalized model is
restricted to the following set of elements and associations between
them, according to our UML profile\footnote{Note that, for the sake
of completeness, we also show \emph{Base} classes which do not
affect summarizability.}:

\begin{itemize}
    \item \emph{Dimension} classes.
    \item \emph{Fact} classes (including fact attributes and degenerate dimensions).
    \item Minimum multiplicity $0$ and maximum $*$ on the side of the \emph{Fact} class in
    fact-dimension associations. Minimum $1$ and maximum $1$ are unusual but permitted,
    however the least restrictive and most common situation is assumed.
    \item Minimum and maximum multiplicity $1$ on the side of the \emph{Dimension} class in fact-dimension associations.
\end{itemize}

We have defined several guidelines to obtain a normalized model.
These guidelines are applied to an initial conceptual MD model
(\emph{source model}) in order to derive a normalized MD model
(\emph{target model}). Each of these guidelines checks the different
kinds of fact-dimension associations in order to create, remove or
modify elements in the source model to obtain a target model which
ensures summarizability, while the expressiveness of the source
model is preserved.

\subsection{Regular Fact-Dimension Relationships}
Regular associations between a fact and a dimension class are those
that have a minimum and maximum multiplicity of $1$ at the end of
the \emph{Dimension} class, regardless of the multiplicities at the
end of the \emph{Fact} class (recall
Table~\ref{a3:tab:FD-multiplicity-classification}). As these
associations do not violate summarizability, they do not require
special treatment. Specifically, this guideline states that regular
fact-dimension relationships in the source model must also appear in
the target model. To this end, for each fact-dimension association
in the source model, it is checked that minimum and maximum
multiplicities in the end of a \emph{Dimension} class are both $1$.
If so, the \emph{Fact} and \emph{Dimension} classes, as well as
their attributes and regular associations, are kept in the target
model exactly as they occur in the source model.


\subsection{Incomplete Fact-Dimension Relationships}
Incomplete associations between a fact and a dimension have minimum
multiplicity of $0$ at the end of the \emph{Dimension} class.
Incompleteness must be eliminated by changing this multiplicity to
$1$ and creating new elements in the target model to keep the
semantic expressiveness of the source model. Therefore, once this
situation is detected in the source model, a new \emph{Fact} class
is created in the target model to store the fact instances that are
not related to the \emph{Dimension} class that causes the
incompleteness. For example, Fig.~\ref{a3:sample_inc1} has an
incomplete relationship between the \emph{Sales} fact and the
\emph{Customer} dimension. The corresponding normalized model is
shown in Fig.~\ref{a3:sample_inc2} where a new
\emph{SalesNoCustomer} fact is created without any association to
the \emph{Customer} dimension. This \emph{SalesNoCustomer} fact has
the fact attributes of the \emph{Sales} fact, and it allows us to
record every \emph{Sales} fact instance that is not related to any
\emph{Customer} dimension instance. Now, in the target model, the
relation between \emph{Sales} and \emph{Customer} is complete, since
the minimum multiplicity at the end of the \emph{Customer} dimension
can be turned into $1$.

\begin{figure}
\begin{center}
\includegraphics[width=0.9\textwidth]{img/dolap08/sample_inc2.png}
\end{center}
\caption{Normalized, complete relationship between \emph{Sales} fact
and \emph{Customer} dimension} \label{a3:sample_inc2}
\end{figure}

%\pagebreak
\subsection{Non-strict Fact-Dimension Relationships}
Non-strictness must be eliminated by turning the maximum
multiplicity $*$ at the end of the \emph{Dimension} class into $1$
and creating the necessary elements in the target model to keep the
semantic expressiveness of the source model. To this end, we note
that non-strictness can occur under two different situations, which
require two different transformations to remove non-strict
fact-dimension relationships. Since these two alternatives have
different meanings, the designer is forced to decide between them.

On the one hand, if the contribution of each dimension instance for
each fact instance is known in advance, then the value of the
measures can be appropriately divided among every dimension
instance. Therefore, the maximum multiplicity $*$ at the end of the
\emph{Dimension} class in the source model can be converted into $1$
in the target model if a degenerate dimension is created in the
\emph{Fact} in order to group the different dimension instances. In
this way, we know the total contribution of a complete group by
considering every individual contribution. For example,
Fig.~\ref{a3:sample_nonstrict1} represents a source model with a
non-strict association between the \emph{Sales} fact and the
\emph{Salesperson} dimension. If we know the amount of individual
sales of every salesperson within a shared sale, then we obtain the
target model of Fig.~\ref{a3:sample_nonstrict2a} by adding a new
degenerate dimension \emph{salespersonGroup} and turning the
multiplicity at the end of the \emph{Salesperson} dimension into
$1$. This degenerate dimension \emph{salespersonGroup} is a grouping
key that allows us to calculate the total sales for a joint sale
from the individual sales of each salesperson.  We note that this
solution is called ``multivalued dimensions''
in~\cite{book/Kimball/DW}, where it is suggested as a solution on
the logical level (see Sect.~\ref{a3:sec:related}). In contrast, in
our approach the additional information about individual sales and
their grouping is expressed explicitly at the conceptual schema
level.

\begin{figure}
\begin{center}
\includegraphics[width=0.8\textwidth]{img/dolap08/sample_nonstr2a_date.png}
\end{center}
\caption{Normalized, strict relationship when individual
contributions are known} \label{a3:sample_nonstrict2a}
\end{figure}

On the other hand, if we do not know the individual contribution of
each dimension instance to the fact, then we are only concerned with
the total contribution. However, data about individual dimension
instances should not be ignored, because they may still be used for
querying purposes. In this case, the maximum multiplicity $*$ at the
end of the \emph{Dimension} class is turned into $1$  by creating a
new \emph{Fact} class in the target model which records all measures
related to individual dimension instances, whilst the measures
related to the group of dimension instances are stored in the old
\emph{Fact} class. Furthermore, a degenerate dimension is created
for each \emph{Fact} class in the target model to express a
correspondence between every individual dimension instance and its
corresponding group, thus enabling the analysis of both facts via
drill-across operations.

For example, assume that we only know the total sales made by a
group of several salespersons in the non-strict association between
the \emph{Sales} fact and the \emph{Salesperson} dimension in
Fig.~\ref{a3:sample_nonstrict1}. To remove non-strictness, we create
a new \emph{SalesIndividual} fact, which is associated with the
\emph{Salesperson} dimension (see the target model of
Fig.~\ref{a3:sample_nonstrict2b}). This new fact contains no
measures, since every sale is recorded with a group of salespersons
in the \emph{Sales} fact. A degenerate dimension
\emph{salespersonGroup} is also created in each fact. These
degenerate dimensions enable the analysis of total sales, at the
same time that we can obtain the individual data of each customer,
which is stored in another fact. In this way, for a joint sale, we
can explicitly show the total sales for a group of salespersons and
recover their individual data by using the new degenerate
dimensions.  This situation is not resolved by ``multivalued
dimensions''~\cite{book/Kimball/DW}, since a ``weighting factor'' is
necessary to identify the individual contributions.

\begin{figure}
\begin{center}
\includegraphics[width=0.7\textwidth]{img/dolap08/sample_nonstr2b_date.png}
\end{center}
\caption{Normalized, strict relationship when only total
contributions are known} \label{a3:sample_nonstrict2b}
\end{figure}

\section{Implementation}
\label{a3:sec:implementation} The guidelines of the normalization
process as described in this paper, have been formally designed by
using the Que\-ry/Vi\-ew/Trans\-for\-ma\-tion (QVT)
language~\cite{OMG/QVT} in order to be automatically performed. This
language is a standard approach for defining formal relations
between MOF-compliant models. Furthermore, QVT is an essential part
of the MDA (Model Driven Architecture)~\cite{OMG/MDA} standard as a
means of defining formal and automatic transformations between
models. The proposed model-transformation architecture has been
implemented in the \textsc{Eclipse} (\url{http://www.eclipse.org/})
development platform, which is a modular open source platform that
can be extended by means of plugins in order to add more features
and new functionality.  We have designed a couple of modules
encapsulated in a unique plugin that provides \textsc{Eclipse} with
capabilities for executing the normalization process described in
this paper. We have defined a \emph{multidimensional module} which
implements the UML profile for MD modeling, and a
\emph{transformation module} which uses the \textsc{ATL} (ATLAS
Transformation Language) engine
(\url{http://www.eclipse.org/m2m/atl/}) for codifying and executing
the mapping patterns (e.g., Fig.~\ref{a3:fig:tool} shows an excerpt
of the ATL code to deal with non-strictness) identified in the QVT
transformations in order to implement the normalization process. By
using these modules, we provide a customized palette tool that
permits to easily make a diagram by using our UML profile for
multidimensional modeling. Additionally, we provide the
corresponding menu extensions in order to launch the corresponding
transformations to obtain a normalized model.  Note that our
implementation naturally allows to deal with complex relationships,
e.g., non-strict \emph{and} incomplete ones, by simply launching the
appropriate transformations one after the other.

\begin{figure}
\begin{center}
\includegraphics[width=\textwidth]{img/dolap08/tool.png}
\end{center}
\caption{Snapshot of our \textsc{Eclipse}-based tool}
\label{a3:fig:tool}
\end{figure}



\section{Related Work}\label{a3:sec:related}
Most of MD modeling approaches only focus on ensuring
summarizability for dimension
hierarchies~\cite{DBLP:journals/tods/HurtadoGM05,DBLP:journals/dke/MalinowskiZ06,DBLP:conf/dawak/MansmannS06,DBLP:conf/er/AkokaCP01}.
Surprisingly, few works address summarizability issues in
fact-dimension relationships and all of them are only concerned with
many-to-many relationships between facts and dimensions, i.e.,
non-strictness, thus ignoring incomplete relationships.

Multivalued dimensions~\cite{book/Kimball/DW} are a first attempt in
this respect, which permit a star schema to have non-strict
relationships between facts and dimensions by means of a bridge
table. This bridge table captures a non-strict fact-dimension
relationship via foreign keys that refer to the tables that
represent the dimension and the fact. These foreign keys also form a
compound primary key for the bridge table. Song et
al.~\cite{DBLP:conf/dmdw/SongRME01} focus on defining several
methods at the level of a relational implementation to improve the
use of a bridge table. They advocate the representation of
many-to-many relationships with correct semantics, maintaining at
the same time the star schema structure by defining six different
approaches. They also give advantages and disadvantages of each
approach and recommendations for their use. However, both
approaches~\cite{book/Kimball/DW,DBLP:conf/dmdw/SongRME01} are
defined at the logical level, which requires a lot of expertise to
model real-world situations in terms of complex schemas. In
particular, those approaches do not explicitly show the different
types of real-world information that might be available at the
conceptual level (e.g., Fig.~\ref{a3:sample_nonstrict2a} and
Fig.~\ref{a3:sample_nonstrict2b} clearly represent two different
real-world situations, which call for different logical
implementation strategies).

Pedersen et al.~\cite{DBLP:journals/is/PedersenJD01} state that
non-strict relationships between facts and dimensions are necessary
in many real case situations, therefore, these relations must be
directly captured in a conceptual model. Nevertheless,
summarizability is tackled at the instance level by modifying the
data in the data warehouse. This may be an unsuitable solution due
to the fact that data sources are huge in data warehouse systems and
performance problems may arise when the required complex exploration
of every stored data instance is done. Furthermore, considering data
instances requires preprocessing tasks (e.g. every time that the
data warehouse is updated, the summarizability must be checked).

The novelty of our approach for solving summarizability in
fact-dimension relationships is the following: (i) we provide a
systematic way to enumerate all cases by using multiplicities at the
conceptual level, which allows us to argue about every case whether
it is problematic or not and to ease the task of designing
real-world situations, (ii) we give mechanisms to design every
situation at the conceptual level by using our UML profile for MD
modeling, and (iii) we provide a normalization process to solve
summarizability at the conceptual level, without using information
from data instances.


\section{Conclusions and Future Work}
\label{a3:sect:conclusions} Ensuring hierarchy-related
summarizability in MD models has been widely tackled by current
research. However, summarizability problems arising from
fact-dimension relationships have been ignored so far. Therefore,
data warehouse designers still face problems when defining
fact-dimension relationships that accurately reflect real-world
situations in a MD model, whilst avoiding summarizability problems.

In this paper, we have described a normalization approach for
ensuring that the implemented MD model will be queried without
summarizability problems arising from fact-dimen\-sion
relationships. Following our approach, in a first step designers may
define fact-dimension associations that ignore summarizability
conditions in a conceptual model by using our UML profile. This
conceptual model reflects real-world situations in an understandable
way. Later, several guidelines can be applied to obtain a normalized
MD model whose fact-dimension relationships do not allow situations
that violate summarizability, thus avoiding erroneous analysis of
data. These guidelines have been implemented in an
\textsc{Eclipse}-based tool by using the QVT language.

%\paragraph{Future work}
Our short-term future work consists of including this normalization
process into our framework for the development of data warehouses
based on MDA~\cite{journals/dke/Mazon2007,journals/dss/Mazon2008}.
We also plan to consider summarizability problems when aggregation
functions are applied to measures from the fact along the different
dimension hierarchies, as suggested
in~\cite{DBLP:conf/er/CherfiP03,DBLP:conf/er/HornerS05}.


%\section{Acknowledgements}
%\label{sect:acknowledgements}  This work has been partially
%supported by the ESPIA project (TIN2007-67078)  from the Spanish
%Ministry of Education and Science, and by the QUASIMODO project
%(PAC08-0157-0668) from the Castilla-La Mancha Ministry of Education
%and Science (Spain). Jose-Norberto Maz{\'o}n is funded by the
%Spanish Ministry of Education and Science under a FPU grant
%(AP2005-1360).

\bibliographystyle{abbrv}
\bibliography{tesis}


%
